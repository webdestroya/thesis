\chapter{Background}

\indent
To better understand our reasoning for creating Calico, it is useful to have some insight into the history of sketching, and in particular the history of sketching in software design. Designers rely on sketching as part of their own thought process.

[FRANDRE]Sketching allows designers to express their ideas in a very fluid and flexible manner. Sketching allows designers to not be hindered by design software that tries to enforce a specific design style. Designers are able to be as meticulous as they wish, and they are not spending time working around the roadblocks created by a structured design system. It is very easily for a designer to cross out a given idea and then immediately create an alternative. Designers typically work with a visual image in their head\cite{todo}, and sketching provides the least amount of friction when trying to put that visual image into the design. This flexibility allows them to focus on the formation of ideas and discussion, without having to worry about how the discussion itself is being documented.  Sketching allows ideas to be more easily viewed, analyzed, and discussed -- much more than would be possible without some representational system. Another benefit afforded by sketching is the ability to view a design from a higher-level or ``bird’s-eye view''. This high-level view allows designers to discover design paths that they would have otherwise overlooked had the idea not been drawn out and understood. 

[ANDRE]Sketching allows designers to create representations of concepts that can be linked together in order to create a very detailed design. For example, flowcharts can be created that show how logic will flow through a program. As another example, database designers can use sketching to informally show foreign key relations when designing a schema. Often in sketching, designers are able to mix very different design styles that a normal software design program would not be allow in the same space. As you can see in [IMAGE], designers at a local company are creating sketches in order to design a software product. You can see how they have linked various components in the system to each other, so that they can create an overview of how the system will function.

In the context of software design, sketching tools prove to be even more useful. These tools allow designers to create several varying solutions in parallel, and then chose the best of these designs to continue designing. Sketching allows software designers to fluidly move focus between various potential ideas in order to contrast designs with others. Software design tools that do not force a specific design structure on the user can encourage a broader consideration of alternative designs, and can greatly improve the eventual design outcome.

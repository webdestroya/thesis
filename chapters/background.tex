\chapter{Background}

To better understand our reasoning for creating Calico, it is useful to have some insight into the history of sketching, and in particular the history of sketching in software design. 

\section{Sketching in Design}

Sketching has proven itself to have a very important role within the design community. 
Designers rely on sketching as part of their own thought process\cite{petre}. Sketching allows the designer a path by which they can go from their original thoughts to a more concrete design. Sketching acts as a very powerful tool that provides designers with fluidity as well as flexibility when translating ideas to drawings\cite{csik}. This flexibility and fluidity allows designers to focus on the ideas and discussion instead of focusing on \emph{how} the design is being recorded. There are many benefits to sketching. In the paragraphs below, we list the benefits that are most important.

First, sketching allows for a tight cycle of drawing, understanding, and reinterpretation. This cycle can be described as having a ``reflective conversation'' with the material\cite{schon}. This reflective conversation can lead to many ``unexpected discoveries'' \cite{suwa} that designers may not have noticed without sketching. Sketching allows designers to formulate new ideas, combine, transform, and also reject ideas

Second, sketching allows designers to quickly externalize their thoughts. This externalization allows for ideas to be viewed, analyzed, and discussed \cite{petre}. Designers may jot down ideas before they forget, or they may create a partial diagram to use as reference when explaining a concept to another designer. These sketches are known as ``thinking sketches'' for their ability to support the thinking process, as well as ``talking sketches'' because of their ability to support the discussion of an idea\cite{ferguson}. 

Third, sketching allows designers to use symbols to create representations of concepts. Once a concept has been reduced to symbols, designers can begin using spacial metaphors such as grouping or scaling to develop new insights\cite{larkin}. This also allows designers to create new notations if no suitable notation exists. Designers use different symbols for different activities, as well as different symbols for different phases within the same activity\cite{goel}. 

\section{Sketching in Software Design}

There have been many studies conducted that have indicated that sketching is used extensively in the field of software design\cite{dekel}. Many studies have noted how design teams utilize sketching and manipulate sketches to support the use of mental imagery during a design session\cite{dekel}\cite{petre}.

Sketching has proven itself to be a valuable tool in the field of design. Studies have shown that sketching enables designers to share thoughts, ground ideas, manipulate concepts, and brainstorm \cite{cherubini}. Diagrams are essential tools for explaining concepts -- both in distributed and co-located teams\cite{yatani}.

In the context of software design, sketching tools prove to be even more useful. Zannier found that tools that encouraged conversations between designers gave way to decisions that considered more alternatives\cite{zannier}. Software design tools that do not force a specific design structure on the user can encourage a broader consideration of alternative designs, and can greatly improve the eventual design outcome\cite{zannier}. These tools allow designers to create several varying solutions in parallel, and then chose the best of these designs to continue designing. Software designers often sketch as a natural extension of the thought process used during the design phase to view more than a single solution simultaneously\cite{petre}. Sketching allows designers to have flexibility while they explore design problems. Sketching allows software designers to fluidly move focus between various potential ideas in order to contrast designs with others. It allows them to go from abstract thoughts to concrete ideas. Petre noted that sketching allows designers to shift focus among concepts in order to easily compare them with one another\cite{petre}. Petre also noticed that younger design sketches tend to be lacking in detail, but as the design grows and matures, more detail may be added to the design.

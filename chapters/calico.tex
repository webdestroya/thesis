\chapter{Calico}

% [anticpated/actual features]
% [purpose, how does it work]
% [supports designers in sketching]

\section{Canvas and Grid}

\begin{figure}[htb]
\centering
\includegraphics[width=0.8\textwidth]{grid.jpg}
\caption{The grid view within Calico}
\label{fig:grid}
\end{figure}
%The grid is the focal point of any session in Calico.
%It shows the various canvases that users may interact with in a given session.
%Users may perform various operations on canvases from the grid, such as duplicating or clearing individual cells.
%The grid also gives a clear overview of the designs that are happening in a session.

\todo{From Andre}
Calico as a drawing tool, is centered on the metaphor of canvases, with the grid acting as the focal point.
It has a mulitude of canvases organized in a grid. [See Figure \ref{fig:grid}]
Users can draw in each of these canvases.
Each of these canvases can contain sketches, writing, anything that the designer desires.
Because these canvases are organized in a grid, users are able to move from one canvas to another in one of two ways: They can switch to a ``grid view'' that gives them an overview of all the cell, or they can use the white navigation tabs located on the side that will allow them to jump to the adjacent canvas.
The gray tabs allow the designer to ``branch off'' and create a copy of the current canvas, but in another cell, so that the two can be easily compared.

\section{Gestures}

\section{Scraps}
Scraps in Calico can be thought of as ``scraps of paper'' that one would place on a desk or on a white board.
Scraps can be easily relocated to different parts of the screen, or even other canvases.
Scraps can be stacked on top of each other and then treated as a unit or group.
By treating scraps as if they were pieces of paper, we [make it easy to understand the manipulation], as designers can easily relate Calico to their current design 

\section{Palette}
The palette in Calico provides users with a ``drawer'' that can easily be used to store commonly used shapes and artifacts.
The palette can be synchronized across sessions so that other users in the session can share the same palette.
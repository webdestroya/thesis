\chapter{Conclusion}
My work with Calico was geared toward solving the three problems found in the previous version of Calico. The first was the instability of Calico. Continual crashes and data loss plagued early versions of Calico. To solve this, we redesigned Calico to use a central server that would act as a stable storage system to prevent data-loss. The second problem was the lack of support for collaboration. Our decision to move to a client-server architecture provided us with a solution to the problem of collaboration. Now with a centralized server, collaboration was easily enabled. The third problem was the lack of extensibility within Calico. Our creation of a well defined plugin framework allowed Calico to be easily extended to meet future requirements with minimal change.

Before these issues were resolved, Calico was used almost exclusively by our own design team for internal projects. By solving these problems, we were able to deploy Calico into many different environments such as professional workplaces and classrooms. We successfully tested Calico using a classroom of approximately 30 students working on various design projects, all connected to a single server. After solving these issues, we also deployed Calico to a local software development company. At this local company, Calico was installed for a few weeks where it could be used by designers and developers on a daily basis. This study ended with mixed results, as Calico was not installed using a projector that was secured to the wall. This meant that the touch board continually needed to be recalibrated, so the usage was not as consistent as it was in the classroom.
\chapter{Conclusion}
My work with Calico was geared toward solving the three problems found in the previous version of Calico. The first was the instability of Calico. Continual crashes and data loss plagued early versions. To solve this, we redesigned Calico to use a central server that would act as a stable storage system to prevent data loss. The second problem was the lack of support for collaboration. Our decision to move to a client-server architecture provided us with a solution to the problem of collaboration. Now with a centralized server, collaboration could be enabled by allowing multiple clients to connect and disambiguating concurrent edits to the same elements. The third problem was the lack of extensibility within Calico. Our creation of a well-defined plugin framework allows Calico to now be extended to meet future requirements with minimal change to the core.

Before these issues were resolved, Calico was used almost exclusively by our own design team for internal projects. By solving these problems, we were able to deploy Calico into many different environments such as professional workplaces and classrooms. We successfully tested Calico using a classroom of approximately 30 students working on various design projects, all connected to a single server. We also deployed Calico to a local software development company. At this local company, Calico was installed for a few weeks where it could be used by designers and developers on a daily basis. This study ended with mixed results, as Calico was not installed using a projector that was secured to the wall. This meant that the touch board continually needed to be re-calibrated, so the usage was not as consistent as it was in the classroom.

\section*{Experiences}
From these experiences, however, we learned...
add 2-3 paragraphs about experience

Not sure what should be in here, we never really ``tested'' the new iteration of calico (and I never studied any results). Should I talk about the use in the classroom, and that it was able to support 30+ students working on it at the same time? (I already talked about that in the previous paragraph)

\section*{Future Work}
While the work done to improve Calico was a major stepping stone, there are still many other directions for future work. One direction in particular would be the addition of sessions to Calico. Sessions would allow users to manage multiple server instances from a single administrative interface. This would allow a designer to create a new server that is completely separate from any existing instance (it would not share any canvases). Another feature that would be benefitial would be a subtype of scraps, called Lists. List objects would act just like a normal scrap element with one difference: they would allow content to be ordered within the list. This would allow designers to easily create ordered lists (or unordered) lists that they could easily refer to during development. 
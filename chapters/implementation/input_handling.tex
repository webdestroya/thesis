\section{Input Handling}
% - talk about input handling system (how i know what object you touch)
% when the mouse pressed, locate all scraps that contain the mouse location
% determine the smallest scrap that contains the location
% actions are performed on that
% input handler is then "locked" onto that specific group (so that during the move, we are not trying to calculate over and over)
% unlocked as soon as it is released
% each object knows the list of points (or path) that acts as its bounds.
% Each object can then know if a specific point is within
In an effort to improve the user interaction experience, we felt that the existing input handling system needed to be recreated. The previous version of Calico had relied on the input handler provided by the drawing framework that was used (Piccolo \cite{piccolo}). This became unusable when working with canvases containing hundreds of objects. To improve response time, we needed to create a new input handler that did not rely on the drawing framework - meaning that it could continue to operate even while the drawing framework was busy rendering the display.

Rather than having a specific mouse listener linked to each object on screen, we decided to have a global mouse listener that determines which object is being touched. This means that mouse input only needs to be processed by a single handler, and based on various modes, can intelligently handle interaction with the user. 

Each canvas maintains a list of all objects that are present on screen. Each object also is responsible for maintaining a list of coordinates that form its ``bounds''. Objects then can readily determine if a specific point is contained within its ``bounds''.
Upon receiving mouse input, the input handler iterates through the elements and determines which elements contain the mouse location. This list is then further reduced to return only the \emph{smallest} object that contains the mouse location. This is helpful when handling scraps that have been stacked on top of each other - the parent scrap still contains the mouse location, but clearly the user wants to interact with one of the child elements.

After locating the element that the user is performing actions with, the input handler is ``locked'' to only interact with that element until the mouse has been released. This means that users can move a scrap around the screen (and move the mouse outside the bounds of the selected scrap) and still have the scrap follow the mouse location. Another benefit of locking the input handler on a specific element is that, while performing very intensive operations such as moving an element across the screen, we are not wasting resources to recalculate which element we are interacting with. This is something that was not available in the previous versions of Calico.

The one drawback to using this approach is that the object location and screen coordinate systems needed to be identical. What that means is that the Calico coordinate system is equal to the user's screen resolution. If a user with a larger screen joins a session, they can draw shapes and figures that are outside the viewable area of users with smaller screens. At the time, this was not a problem because all users were running the same resolution, so fortunately this problem rarely occurred.
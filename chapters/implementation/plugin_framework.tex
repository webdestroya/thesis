\section{Plugin Framework}
To improve the extensibility of Calico, a plugin framework was created. This plugin framework allows for new features to be integrated into Calico without needing to directly modify its core code. Plugins subscribe to specific events that they want to be notified about. When one of these events is triggered by Calico, all subscribers to that event are notified, and can choose to perform any action based upon the event information. 

\begin{figure}[h!]
  \centering
  \small
  \verbatiminput{figures/plugin.java}
  \normalsize
  \caption{Example plugin source code}
  \label{code:plugin_file}
\end{figure}

To create a plugin, developers only have to extend a provided abstract plugin class that allows each plugin to register itself with a \texttt{PluginManager} that is responsible for publishing network events and interface events to each plugin (see Figure \ref{code:plugin_file}). Plugins were provided with default ``hooks'' that are called when the plugin is loaded or unloaded. Plugins call a method \texttt{registerNetworkCommandEvents} to subscribe to specific network commands that it was interested in. Another method, \texttt{RegisterPluginEvent}, enables the plugin to subscribe to various interface events that are used for processing user input and clicks. For instance, plug-ins can register to be notified when a scrap or stroke is created. When the server receives the \texttt{SCRAP\_FINISH} event, the plugin will immediately be notified about the scrap, and will be given the identifier corresponding to the scrap. With this information, the plugin can obtain the scrap from the database and can then perform further operations if needed. For example, a plugin can listen for newly created scraps and submit a screenshot of each scrap to a remote service as soon as the user creates the scrap.

\subsection*{Interaction with Core Elements}
We decided to allow plugins to access the core Calico controllers to be able to create, modify, and delete core elements (\texttt{Scrap}s, \texttt{Stroke}s, \texttt{Arrow}s). For example, we created a chat robot that would listen for incoming text messages, and upon receiving a message, creates a new \texttt{Scrap} with the given message as the contents. In another example, we created a plugin that generates an image based on the current canvas and submits this image to an outside service to be saved.

\subsection*{Restrictions}
While plugins were given a great amount of freedom when it came to interacting with the core elements, they are not allowed to modify Calico's interface. Plugins in particular can not create custom versions of \texttt{Scrap} elements or \texttt{Stroke} elements. We decided that we did not want to allow that flexibility. Instead, plugins are given just the freedom to manipulate elements, but not customize them further. Additionally, plugins are not allowed to customize the global interface of Calico -- they are not allowed to add buttons or change the layout of the menus. Again, these were not features we thought necessary for the level of customization that plugins were designed for.
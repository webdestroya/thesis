\chapter{Introduction}

Software design tools offer the user a significant amount of power when it comes to tool support. 
Using these tools, designers can generate various documentation artifacts such as architecture diagrams, flowcharts, relational diagrams.
Tools have been created to support all forms of software design, ranging from object-oriented software design to database layouts to logic flow.
Many tools allow designers to create UML diagrams that can be used to represent a software program. ArgoUML \cite{argouml} is one such tool that has been downloaded thousands of times.
Other general tools such as Microsoft Visio \cite{visio} allow designers to create diagrams such as database layouts and flowcharts. 
Architectural diagram tools such as ArchStudio \cite{archstudio} allow designers to create a representation of a program's architecture and connections.
Many other tools exist; there is no shortage of tools to help designers create a broad range of different kinds of designs.


Despite these tools clearly helping in certain aspects of the design process, designers do not always use these tools.
When faced with a design problem, software designers will not necessarily use these tools from the start. 
As observed by Cherubini \cite{cherubini} and Petre \cite{petre}, many times designers will instead turn to a whiteboard or pen-and-paper in order to work through the design problem.


There are many reasons that designers turn to informal tools to help them during the design process. 
A few key benefits of using an informal design tool are listed below.
\begin{itemize}\itemsep1pt
\item 
Designers are free from the constraints of a formal design notation \cite{wong}. 
Using an informal design tool means that a designer is not required to create a formal design document. 
There is nothing to force them to adhere to UML, ERD, or any other formal notation. 
An informal tool allows the designer to easily sketch out a design in whatever notation works best for them. 
Designers want to work through a design quickly, without being bogged down by the details of the tool or the notation.
Formal design tools enforce strict adherence to a design notation.
When designers are forced to stop and create elements that adhere to the specific notation, they are wasting time that could be used to improve the design.

\item 
Sketching allows designers to keep up with their thinking process \cite{petre}. 
Designs can change at a fairly quick pace, and the tools that support designers need to be able to function effectively at this pace.
Formal design tools can be slow to create a design with, and this impedes the design process. 

\item 
Whiteboards and pen-and-paper are ideal mediums for collaboration.
They allow multiple designers to work at the same time on the same design.
Designs can be quickly created and modified by different designers at the same time, which helps improve the design process.
\end{itemize}

The affinity that designers have for using a whiteboard or pen-and-paper has been recognized across many different disciplines of design \cite{goel, cherubini}. 
Sketching plays a crucial and universal role in all fields of design. 
Despite this ubiquitous and advantageous role, a question arises whether sketching may be supported in a better way.
Using a whiteboard or pen-and-paper only allows a designer to add and delete content, nothing more.
What if there was software that was specifically designed to support the informal design process, rather than supporting the formal design process? 
What benefits would this software be able to provide designers?


This question is what led to the development of Calico \cite{calico2}. 
Calico is a software-sketching tool that aims to support designers by helping them easily draft potential software systems. 
Calico was designed to be used with a multi-touch whiteboard and projector so that it can replicate the role of a traditional whiteboard. 
The goal was to build upon the ease-of-use that a whiteboard provides designers, while at the same time providing options and benefits that a traditional whiteboard system lacks, such as the ability to revert to previous designs, or the ability to branch off designs. 
Calico has a number of features, such as ``scraps,'' gestures, and the grid, that were explicitly designed to support the informal design process.
Calico, is very similar to a traditional whiteboard in that it allows the user to sketch freely, but does not suffer from the disadvantages that a whiteboard does.


However, the early versions of Calico exhibited a number of problems when it came to supporting sketchy and improving upon the traditional whiteboard experience tools.
\begin{itemize}\itemsep1pt
\item 
Distributed designers were not well supported.
The network system in Calico was sluggish, and as a result, distributed designers were not able to effectively collaborate with one another. The current network also did not function behind a firewall, so designers could only collaborate with designers on the local network.

\item 
Collaboration in early versions of Calico was not supported. 
Calico only supported a single user on a single ``canvas'', and designers were not able to work collaboratively on designs. 
As shown earlier, the ability to work collaboratively is one of the advantages of using a whiteboard or pen-and-paper, and without collaborative support, Calico was not equal to a whiteboard.

\item
It was very difficult to add new functionality, which prevented new features and plugins from being added. 

\end{itemize}
My research was about addressing these problems in order to improve Calico.


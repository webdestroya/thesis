\chapter{Introduction}

% 24:50
% Lots of software design tools have been proposed and put to use over the years,
% Early tools like ___, general tools like visio, or specialized tools like argouml, archstudio.
% these tools are all very popular and have made great improvements in the area of software design.
% many tools have had a significant impact, argouml is open source and has been downloaded and used by thousands 
Software design tools offer the user a significant amount of power when it comes to tool support. 
Designers can generate diagrams and other documentation artifacts using these tools.
Tools have been created to support all forms of software design ranging from object-oriented software design to database layouts to logic flow.
Many tools allow designers to create UML diagrams that can be used to represent a software program. ArgoUML\cite{argouml} is one such tool that has been downloaded thousands of times.
Other general tools such as Microsoft Visio\cite{visio} allow designers to create diagrams such as database layouts and flowcharts. 
Architectural diagram tools such as ArchStudio\cite{archstudio} allow designers to create a clear representation of a program's architecture and connections.
Many of these tools have had a significant impact in the field of design, and many different tools have been proposed and created over the years.
In the area of software design, there is no shortage of tools to help designers create a representation of a software component.

% PARA
% despite the availability of these tools, which clearly help in certain aspects of the design process
% when faced with a design problem, its been observed that software designers will not necessarily use these tools right away
% but first they will go to the whiteboard or use pen/paper to work through the design problem
% #CITE work (marian petre, andy ko) that supports this
Despite the usefulness of these tools which clearly helps in certain aspects of the design process, designers do not always use these tools.
When faced with a design problem, software designers will not necessarily use these tools from the start. 
As observed by Ko\cite{andy_ko} and Petre\cite{petre}, many times designers will turn to a whiteboard or pen-and-paper in order to work through the design problem.

% PARA
% its interesting to observe why designers use a whiteboard/ pen&paper. what are the benefits that they get?
% - developers are free from the constraints of the notation
% - they want to work through a design quickly, they cant be bogged down by the details of the tool or the notation
% - sketching allows them to keep up with their thinking process
% - whiteboard is an ideal medium to collaborate, mutliple people can work at a board at the same time.
It is interesting to observe why designers turn to a whiteboard or pen-and-paper and are reluctant to utilize powerful design tools.
There are many reasons that designers turn to informal tools to help them during the design process. 
A few key benefits of using an informal design tool are listed below.
\begin{itemize}\itemsep1pt
\item 
Designers are free from the constraints of a formal design notation. 
Using an informal design tool means that a designer is not required to create a formal design document. 
There is nothing to force them to adhere to UML, ERD, or any other formal notations. 
An informal tool allows the designer to easily sketch out a design in whatever notation works best for them.

\item 
Designers want to work through a design quickly, without being bogged down by the details of the tool or the notation.
Formal design tools enforce strict adherence to a design notation.
When designers are forced to stop and create elements that adhere to the specific notation, they are wasting time that could be used to improve the design.

\item 
Sketching allows designers to keep up with their thinking process. 
Designs can change at a fairly quick pace, and the tools that support designers need to be able to function effectively at this pace.
Formal design tools can be slow to create a design with, and this impedes the design process. 

\item 
Whiteboards and pen-and-paper are ideal mediums to collaborate using.
This allows multiple designers to work at the same time on the same design.
Designs can be quickly created and modified by different designers at the same time, which helps improve the design process.
\end{itemize}

% PARA
% despite this preference for whiteboard and a real informal nature
% one might ask: can we support this informal role of design in a better way?
% whiteboard only allows you to add and delete content, nothing more
% what if there was software that was designed to support the informal design process, rather than software to support the formal design process. -- what would that look like?
The affinity that designers have for using a whiteboard or pen-and-paper has been recognized across many different disciplines of design. 
Sketching plays a very crucial, and very universal role in all applications of design. 
How can we support this informal role of design in a better way?
Using a whiteboard or pen-and-paper only allows a designer to add and delete content, nothing more.
What if there was software that was specifically designed to support the informal design process, rather than supporting the formal design process? 
What benefits would this software be able to provide designers with?

% PARA
% this question is what led to the development of calico
% a sketch based software design tool meant to be used on a whiteboard
% calico has a number of features that were explicitly designed (scraps, gestures, grid)
This question is what led to the development of Calico. 
Calico is a software-sketching tool used by designers to help them easily draft potential software systems. 
Calico was designed to be used with a multi-touch whiteboard and projector so that it can replicate the role of a traditional whiteboard. 
The goal was to build upon the ease-of-use that a whiteboard provides designers, while at the same time providing options and benefits that a traditional whiteboard system lacks, such as the ability to revert to previous designs, and the ability to branch off designs. 
Calico has a number of features, such as ``scraps,'' gestures, grid, that were explicity designed to support the informal design process.
Calico can be used to be overcome the limitations of a traditional whiteboard, while not suffering the disadvantages of a structured design tool. 
Calico, is very similar to a traditional whiteboard in that it allows the user to sketch freely, but does not suffer from the disadvantages that a whiteboard does.

% PARA
% however, calico as it was exhibited a number of problems when it comes to supporting this design process
% - it wasnt stable
% - it wasnt collaborative
% - it was missing X, Y, ...
% my research was about addressing those problems
However, the early versions of Calico exhibited a number of problems when it came to supporting this design process and improving upon available informal design tools.
\begin{itemize}\itemsep1pt
\item 
Stability proved to be a serious issue. 
Early users often encountered crashes with Calico that would cause them to lose any work they had on screen. 
Calico needed to be stable enough to remain active for long periods of time.

\item 
Collaboration in early versions of Calico was not supported. 
Calico only supported a single user on a single ``canvas'', and designers were not able to work collaboratively on designs. 
As shown earlier the ability to work collaboratively is one of the advantages of using a whiteboard or pen-and-paper, and without collaborative support, Calico was not equal to a whiteboard.

\item
It was very difficult to add new functionality, which prevented new features and plugins from being added. 

\end{itemize}
My research was about addressing these problems in order to improve Calico.


%%%%%%%%%%%%%%


% Historically when designers are constructing their designs, a whiteboard is used to sketch and modify their ideas.
% This free flowing process of sketching and manipulating changes allows for an open ended thought process whereby the designer has a visualization of what was previously only in his thoughts. It also allows for expansion on ideas and for peers to freely edit, annotate, and revise the original idea. At the end of this design process, the project's design must be saved to a formal design notation. Unfortunately, it is this very process of formal documentation that often impedes the free-flow of thought and reflection on the design itself. When working at the whiteboard, developers typically draw very informal table-napkin type drawings that are used to quickly build a solution to a given problem. Whiteboards are excellent tools for this, as they provide very little resistance to quick sketching. 

% Formal design notations lack the ability to track these rough sketches and original thought processes. These formal design notations also lack the ability for the free-flow of exchange and flexibility when several designers are working simultaneously on the same project. Often a single designer will create a design, and then must wait while another designer modifies the design. The interaction between the two designers and the opportunity for immediate feedback is lost in this ever-changing environment. The free-flow of ideas and exchanges that take place on a whiteboard with sketching are often inhibited by the very rigor that is required of a structured document.

% The affinity that designers have for using a whiteboard has been recognized across many different disciplines of design. Sketching plays a very crucial, and very universal role in all applications of design. 

% For example, architects may make very informal sketches of a building before they ever begin drafting blueprints of their design.
% Automobile designers can create informal drawings of their ideas before creating a formal design of a car.
% Sketching can be seen in all design disciplines, and is a very powerful process that can be used to effectively guide a final design.

% A platform that allows sketching and the free-flow of design allows the user to be free from impediments such as adherence to a formal design notation. This allows the free-flow of ideas and increased creativity during the design process.

% Current software design tools offer the user a significant amount of power when it comes to tool support. Designers can generate diagrams and other documentation artifacts using these tools. However, these tools do not seem to exist when it comes to sketching. Designers need a tool that is flexible and fluid like a whiteboard, but powerful enough to be useful to designers as a tool to thorougly capture original work.  

% Calico is a software-sketching tool used by designers to help them easily draft potential software systems. Calico was designed to be used with a multi-touch whiteboard and projector so that it can replicate the role of a traditional whiteboard. The goal was to build upon the ease-of-use that a whiteboard provides designers, while at the same time providing options and benefits that a traditional whiteboard system lacks, such as the ability to revert to previous designs, and the ability to branch off designs. Calico can be used to be overcome the limitations of a traditional whiteboard, while not suffering the disadvantages of a structured design tool. Calico, is very similar to a traditional whiteboard in that it allows the user to sketch freely, but does not suffer from the disadvantages that a whiteboard does. 

% When designing a sketching system several considerations were essential to include in the design. The first was the need for the system to be distributed. A traditional whiteboard does not allow for collaboration between designers in different locations. By implementing a distributed, networked design system, Calico enables designers to collaborate with one another not only across an individual workplace, but across the globe -- something a traditional whiteboard is incapable of achieving. Calico was created to allow any number of clients from various locations to interact on the same design space.

% Extensibility allows the design process to change course over several iterations while still being able to document and track these changes. Flexibility is essential for a designer. The ability to rapidly add new features makes experimenting with new ideas relatively simple. Calico was created with this requirement of malleability in mind. Users need to be able to easily add plugins and new features.
% \todo{maybe expand on plugins?} 
% The requirement in the design of Calico to include extensibility allows for easy manipulation to the design and does not bind the designer to the tedious process of formally documenting their design.

% Finally, reliability is essential for any tool so that the designer is not impeded by constant interruptions or crashes. 
% \todo{we wanted this to be able to be run on a project indefinitely, so crashes were bad} 
% The system must maintain the ability to stay online during the entire design process. Whiteboards do not ``crash'', and Calico needed to replicate that stability. 
% \todo{expand on stability}
% By using a centralized server, the system would have a significant increase in stability. Reliability is necessary for designers to be able to continuously use the system.

% Calico was created so that designers could freely expand on ideas through the use of a whiteboard. Designers can collaborate and make revisions to existing designs simultaneously. The power and flexibility offered by Calico allows users to more efficiently bring their designs to fruition. By using a whiteboard and projector, this sketching tool replaces the traditional whiteboard. By including a distributed system designers can work from any location. Extensibility is essential for a designer to add new features whenever necessary. The reliability of the system must be maintained throughout the design process. Calico combines the flexibility of a whiteboard with the technology of modern software to support designers during the design process.


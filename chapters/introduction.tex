\chapter{Introduction}


Historically when designers are constructing their designs, a whiteboard is used to sketch and modify their ideas.
This free flowing process of sketching and manipulating changes allows for an open ended thought process whereby the designer has a visualization of what was previously only in his thoughts. It also allows for expansion on ideas and for peers to freely edit, annotate, and revise the original idea. At the end of this design process, the project's design must be saved to a formal design notation. Unfortunately, it is this very process of formal documentation that often impedes the free-flow of thought and reflection on the design itself. When working at the whiteboard, developers typically draw very informal table-napkin type drawings that are used to quickly build a solution to a given problem. Whiteboards are excellent tools for this, as they provide very little resistance to quick sketching. 

Formal design notations lack the ability to track these rough sketches and original thought processes. These formal design notations also lack the ability for the free-flow of exchange and flexibility when several designers are working simultaneously on the same project. Often a single designer will create a design, and then must wait while another designer modifies the design. The interaction between the two designers and the opportunity for immediate feedback is lost in this ever-changing environment. The free-flow of ideas and exchanges that take place on a whiteboard with sketching are often inhibited by the very rigor that is required of a structured document.

%3
The affinity that designers have for using a whiteboard has been recognized across many different disciplines of design. Sketching plays a very crucial, and very universal role in all applications of design. 
% Examples
%\todo{2-3 examples of sketching in a design. (3-5 sentences)}
For example, architects may make very informal sketches of a building before they ever begin drafting blueprints of their design.
Automobile designers can create informal drawings of their ideas before creating a formal design of a car.
Sketching can be seen in all design disciplines, and is a very powerful process that can be used to effectively guide a final design.
% /examples
A platform that allows sketching and the free-flow of design allows the user to be free from impediments such as adherence to a formal design notation. This allows the free-flow of ideas and increased creativity during the design process.

%4
Current software design tools offer the user a significant amount of power when it comes to tool support. Designers can generate diagrams and other documentation artifacts using these tools. However, these tools do not seem to exist when it comes to sketching. Designers need a tool that is flexible and fluid like a whiteboard, but powerful enough to be useful to designers as a tool to thorougly capture original work.  

%5
Calico is a software-sketching tool used by designers to help them easily draft potential software systems. Calico was designed to be used with a multi-touch whiteboard and projector so that it can replicate the role of a traditional whiteboard. The goal was to build upon the ease-of-use that a whiteboard provides designers, while at the same time providing options and benefits that a traditional whiteboard system lacks, such as the ability to revert to previous designs, and the ability to branch off designs. Calico can be used to be overcome the limitations of a traditional whiteboard, while not suffering the disadvantages of a structured design tool. Calico, is very similar to a traditional whiteboard in that it allows the user to sketch freely, but does not suffer from the disadvantages that a whiteboard does. 
% no need to define reason for name - it's probably something silly

When designing a sketching system several considerations were essential to include in the design. The first was the need for the system to be distributed. A traditional whiteboard does not allow for collaboration between designers in different locations. By implementing a distributed, networked design system, Calico enables designers to collaborate with one another not only across an individual workplace, but across the globe -- something a traditional whiteboard is incapable of achieving. Calico was created to allow any number of clients from various locations to interact on the same design space.

Extensibility allows the design process to change course over several iterations while still being able to document and track these changes. Flexibility is essential for a designer. The ability to rapidly add new features makes experimenting with new ideas relatively simple. Calico was created with this requirement in malleability in mind. Users need to be able to easily add plugins and new features.
\todo{maybe expand on plugins?} 
The requirement in the design of Calico to include extensibility allows for easy manipulation to the design and does not bind the designer to the tedious process of formally documenting their design.

Finally, reliability is essential for any tool so that the designer is not impeded by constant interruptions or crashes. 
\todo{we wanted this to be able to be run on a project indefinitely, so crashes were bad} 
The system must maintain the ability to stay online during the entire design process. Whiteboards do not ``crash'', and we hoped to replicate that stability. 
\todo{expand on stability}
By using a centralized server, the system would have a significant increase in stability. Reliability is necessary for designers to be able to continuously use the system.

% One of the benefits of having a digital design interface was that the user could easily revert to any of their previous designs at the touch of a button.

CONCLUSION


% Is there another word we can use besides sketching? 
% Dont use WE

%%%%%%%%%%%%%%%%%%%%%%%%%%%%%%%%%%%%%%%%%%%%%%%%%%%%%%%%%%%%%%%%%%%%%%%%%%%%%%%%%%%%%%%%%%%%%%%%%%%%%%%%


% [Sketching in software design]
% [what limits current systems from doing what we do]
% [with respect to the needs we have, they fall short]
% [how does calico overcome those limitations]

% Formal design notations are best used for documenting a system, and do not work well when designing. This is the very reason that many designers turn to a whiteboard in order to sketch their designs. A platform that allows for sketching allows the designer to work without any restrictions at all [ON WHAT?] on designers and the drawings they are able to easily create. Designers are then able to freely manipulate their drawings without the burden of a structured design document. The software is able to aid the designer in creation of their drawing, but does not prevent the flexibility that a traditional whiteboard provides.  

% When working at the whiteboard, developers typically draw very informal table-napkin type drawings that are used to quickly build a solution to a given problem. Whiteboards are excellent tools for this, as they provide very little resistance to quick sketching. 

% The affinity that designers have for using a whiteboard has been recognized across many different disciplines of design. Sketching plays a very crucial, and very universal role in all applications of design. [this seems so weak?]

% Current software design tools offer the user a great amount of power when it comes to tool support. Designers can generate diagrams and other documentation artifacts using these tools. However, these tools do not seem to exist when it comes to sketching. Designers need a tool that is flexible and fluid like a whiteboard, but powerful enough to be useful to designers as a tool.  

% Calico is a software-sketching tool used by designers to help them easily draft potential software systems. Calico was designed to be used with a multi-touch whiteboard and projector so that it can replicate the role of a traditional whiteboard. We hoped to build upon the ease-of-use that a whiteboard provides designers, but while at the same time providing options and benefits that a traditional whiteboard system lacks. [I feel like I say this over and over] We hope that Calico can be used to be overcome the limitations of a traditional whiteboard, while not suffering the disadvantages of a structured design tool. While using a whiteboard is very fluid and unrestricted, a whiteboard does not provide any help to a designer -- it is identical to pen-and-paper drawings. 

% Our research has been about designing Calico, which is very similar to a traditional whiteboard in that it allows the user to sketch freely, but does not suffer from all of the disadvantages that a whiteboard does. One advantage Calico has over a traditional whiteboard is collaboration. Being a distributed, networked design system, Calico can enable designers to collaborate with one another across the globe – something that a traditional whiteboard could never achieve. [what do you think is the best ``story'' to tell here?]

% [RANDOM PARA]To improve the existing version of Calico, we decided to vastly improve the architecture in order to natively support collaboration. We had a few requirements that we wanted to satisfy with the new architecture. The first was the need for the system to be distributed. Originally, we wanted to support two boards that were placed next to each other and just provide a link between those two boards. However, in the end we decided against that and chose to create a system that would allow any number of clients, from various locations to interact on the same space. The second requirement was extensibility. We wanted to easily be able to add plugins and new features to Calico. The previous version proved to be very difficult to add new features, so our goal was to solve that problem, and easily allow new things to be added. The final requirement was reliability. We needed to have a system that was able to stay online, and retain the drawings that we created. Old versions of calico would crash, and then all designs would be lost. By having a centralized server, we hope that the server would be able to outlive client crashes and problems, and could be much more stable because it was not handling any graphical interface. With our new architecture we were able to continually experiment with new features. We could see which features were working well, and which ones were rarely used and could then be removed. The relative ease to adding new features made experimenting with new ideas very easy.

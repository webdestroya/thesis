\chapter{Objectives}

% PARA
% when wefirst examined calico, there were a number of problems
% LIST SOME PROBLEMS
% this caused us to take a step back an dlook at the current arch, current design, and current set of requirements and take a holistic view of the project
When we first examined early versions of Calico, there were a number of problems that existed. 
Some of the problems that existed were: lack of native support for distributed users,
collaboration between designers was not natively supported,
lack of a generic plugin interface to add new features, 
and sluggish input handling. 
In addition to these problems, there were also a number of additions that we felt would be beneficial to future users of Calico.
These shortcomings and possible enhancements caused us to take a step back and reexamine the current architecture, current design, and current set of requirements that we had for the original version of Calico described in Section 3.


% when rethinking our architecture, we came up with the following objectives
% - must be natively distributed ::: original calico was built and then collaboration was added on later which did not work. we wanted to design an arch that was distributed from the start
% END
When rethinking the architecture of Calico, we decided on the following objectives for the new version:
\begin{itemize}\itemsep1pt
\item 
\textbf{Must natively support distributed designers.}
\newline
The original version of Calico was created as a single-user application that was not distributed at all. In a later version of Calico, networking support was added as a ``patch.'' It was not natively supported, because of which the overall performance suffered.
We wanted to design an architecture for Calico that was distributed from the start.

\item
\textbf{Must natively support collaboration between designers.}
\newline
As described in the previous paragraph, the original version of Calico was not collaborative. 
Collaboration was added later, but it was not able to properly deal with conflicts between designers modifying the same object at the same time. 
Given that the new version of Calico was to be distributed from the start, we needed to ensure that collaboration was also natively supported.
Designers needed to be able to work in parallel within the same canvas and same objects, and conflict resolution was a clear objective in the new version of Calico. 

\item
\textbf{Must provide a generic interface for external plugins.}
\newline
The original version of Calico provided no interfaces for external plugins. 
Adding new features to the original Calico always required changing Calico itself.
Due to this fact that plugins were being created as patches to the core, there was no way to easily disable or enable specific features - users were forced to run with the full feature-set. 
With the next iteration of Calico, we wanted to prevent this by providing an easy-to-use plugin system that would allow developers to create plugins that could be more easily added to Calico, and could be enabled or disabled on demand.

\end{itemize}


In addition to solving issues that we noticed in Calico, we also wanted to use this opportunity to optimize existing design choices that were used.
We identified a number of features that we decided should be implemented in the new version of Calico.
While these features were not fixing critical bugs, we hoped that these additions would ensure that Calico would be a viable product in the future. 
The features that we decided to add were:
\begin{itemize}\itemsep1pt
\item 
\textbf{Should provide an easy-to-use administration interface.}
\newline
Calico was on its way to being used in many different places, and we needed an easy way to manage the servers. 
The previous version of Calico was a standalone peer-to-peer service, so there was no administration interface provided to the user.
We decided that the new Calico server should provide a web-based interface that would be easy for users to interact with, and would allow users to perform all the necessary administrative functions.

\item 
\textbf{Network usage should be optimized to reduce congestion.}
\newline
The previous version of Calico required a very fast connection when working with other users. If a user was on a slower connection, then they would see updates well after they were made, and this delay made Calico almost unusable in any location other than local Ethernet.
We wanted to reduce the bandwidth that was required to use Calico, in the hopes that designers would be able to collaborate remotely over the Internet -- even in other countries.

\item 
\textbf{Input handling response should be improved.}
\newline
One goal of Calico is to replace a traditional whiteboard with a digital version. 
In order to accomplish this, we needed to reduce the input-response delay as much as we could. 
If a user were to quickly scribble some text on the canvas, Calico had to be able to keep up without lagging. 
Previous versions of Calico had a complex event handling architecture, which meant that the faster a designer drew on the screen, the slower was the response of Calico. This lag was not acceptable, and we hoped that an overhaul of the input handling system would allow Calico to keep pace with all sketching based input.

\end{itemize}


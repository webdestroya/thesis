% This is a template for Ph.D. dissertations in the UCI format.

% All fonts, including those for sub- and superscripts, must be 10 points or larger.
% Recommended sizes are 14-point for chapter headings, 12-point for the main body of text
% and figure/table titles, and 10-point for footnotes, sub- and super-scripts, and text in
% figures and tables.
\documentclass[12pt,fleqn]{ucithesis}

\usepackage{amsmath}
\usepackage{array}
\usepackage{bm}
\usepackage{boxedminipage}
\usepackage{graphicx}
%\usepackage{natbib}
%http://merkel.zoneo.net/Latex/natbib.php
\usepackage[numbers]{natbib}
\usepackage{path}
\usepackage{psfrag}
\usepackage{relsize}
%\usepackage{subfigure}
%\usepackage{subfig}
\usepackage{todonotes}
\usepackage{bytefield}
\usepackage{url}
\usepackage{verbatim}
\usepackage{caption}
\usepackage{subcaption}
\usepackage{listings}

% plainpages=false fixes the "duplicate ignored" error with page counters
% Set pdfborder to 0 0 0 to disable colored borders around PDF hyperlinks
\usepackage[plainpages=false,pdfborder={0 0 0}]{hyperref}



\graphicspath{{figures/}}

% Uncomment the following line to enable Unicode support. This will allow you
% to enter non-ASCII characters (such as accented characters) directly without
% having to use LaTeX's awkward escape syntax (e.g., \'{e})
% NOTE: You may have to install the ucs.sty package for this to work. See:
% http://www.unruh.de/DniQ/latex/unicode/
% \usepackage[utf8x]{inputenc}

\begin{document}


\thesistitle
{
  Improving the Architecture of Calico
}

\degreename{Master of Science}

% Use the wording given in the official list of degrees awarded by UCI:
% http://www.rgs.uci.edu/grad/academic/degrees_offered.htm
\degreefield{Information and Computer Science}

% Your name as it appears on official UCI records.
\authorname{Mitchell Ryan Dempsey}

% Use the full name of each committee member.
\committeechair{Professor Andr\'{e} van der Hoek}
\othercommitteemembers
{
  Professor James A. Jones\\
  Professor Richard N. Taylor
}

\degreeyear{2012}

\copyrightdeclaration
{
  {\copyright} {\Degreeyear} \Authorname
}

% If you have previously published parts of your manuscript, you must list the
% copyright holders; see Section 3.2 of the UCI Thesis and Dissertation Manual.
% Otherwise, this section may be omitted.
% \prepublishedcopyrightdeclaration
% {
%   Chapter 4 {\copyright} 2003 Springer-Verlag \\
%   Portion of Chapter 5 {\copyright} 1999 John Wiley \& Sons, Inc. \\
%   All other materials {\copyright} {\Degreeyear} \Authorname
% }

% The dedication page is optional.
% \dedications
% {
%   To my parents...
% }

\acknowledgments
{
  I would like to thank my advisor, Professor Andr\'{e} van der Hoek. Without his help and guidance my thesis would not be possible.
  I want to thank the members of the Software Design and Collaboration Lab: Nick Mangano, Nicolas Lopez, Gerald Bortis, Alex Baker, Tiago Proenca, and Nilmax Moura. I truly enjoyed my time spent with them, and greatly appreciate the help they have provided over the years.
}

% max 250 words
\thesisabstract
{
  Calico is a software design sketching tool that aims to support software designers by helping them work through a design problem by sketching potential solutions. 
  Calico was designed to be used with an interactive whiteboard so that it can replicate the feel of a traditional whiteboard while at the same time enabling advanced functionality. 

  There were three main problems with the original version of Calico: 
  (1) distributed designers were not well supported, 
  (2) collaboration was not natively supported,
  and (3) it was very difficult to add new functionality.
  This thesis examines these problems and presents various solutions:
  (1) creating a performance-based network architecture,
  (2) centering Calico around collaboration,
  and (3) creating a plugin framework to easily add new functionality.

  Prior to this work, Calico was unable to meet the demands of designers in the field. By resolving these problems, Calico was able to be deployed both in-house in our design lab, as well as in Informatics 122 (a software design course).
}

\preliminarypages

\chapter{Introduction}


Historically when designers are constructing their designs, a whiteboard is used to sketch and modify their ideas.
This free flowing process of sketching and manipulating changes allows for an open ended thought process whereby the designer has a visualization of what was previously only in his thoughts. It also allows for expansion on ideas and for peers to freely edit, annotate, and revise the original idea. At the end of this design process, the project's design must be saved to a formal design notation. Unfortunately, it is this very process of formal documentation that often impedes the free-flow of thought and reflection on the design itself. When working at the whiteboard, developers typically draw very informal table-napkin type drawings that are used to quickly build a solution to a given problem. Whiteboards are excellent tools for this, as they provide very little resistance to quick sketching. 

Formal design notations lack the ability to track these rough sketches and original thought processes. These formal design notations also lack the ability for the free-flow of exchange and flexibility when several designers are working simultaneously on the same project. Often a single designer will create a design, and then must wait while another designer modifies the design. The interaction between the two designers and the opportunity for immediate feedback is lost in this ever-changing environment. The free-flow of ideas and exchanges that take place on a whiteboard with sketching are often inhibited by the very rigor that is required of a structured document.

%3
The affinity that designers have for using a whiteboard has been recognized across many different disciplines of design. Sketching plays a very crucial, and very universal role in all applications of design. 
% Examples
%\todo{2-3 examples of sketching in a design. (3-5 sentences)}
For example, architects may make very informal sketches of a building before they ever begin drafting blueprints of their design.
Automobile designers can create informal drawings of their ideas before creating a formal design of a car.
Sketching can be seen in all design disciplines, and is a very powerful process that can be used to effectively guide a final design.
% /examples
A platform that allows sketching and the free-flow of design allows the user to be free from impediments such as adherence to a formal design notation. This allows the free-flow of ideas and increased creativity during the design process.

%4
Current software design tools offer the user a significant amount of power when it comes to tool support. Designers can generate diagrams and other documentation artifacts using these tools. However, these tools do not seem to exist when it comes to sketching. Designers need a tool that is flexible and fluid like a whiteboard, but powerful enough to be useful to designers as a tool to thorougly capture original work.  

%5
Calico is a software-sketching tool used by designers to help them easily draft potential software systems. Calico was designed to be used with a multi-touch whiteboard and projector so that it can replicate the role of a traditional whiteboard. The goal was to build upon the ease-of-use that a whiteboard provides designers, while at the same time providing options and benefits that a traditional whiteboard system lacks, such as the ability to revert to previous designs, and the ability to branch off designs. Calico can be used to be overcome the limitations of a traditional whiteboard, while not suffering the disadvantages of a structured design tool. Calico, is very similar to a traditional whiteboard in that it allows the user to sketch freely, but does not suffer from the disadvantages that a whiteboard does. 
% no need to define reason for name - it's probably something silly

When designing a sketching system several considerations were essential to include in the design. The first was the need for the system to be distributed. A traditional whiteboard does not allow for collaboration between designers in different locations. By implementing a distributed, networked design system, Calico enables designers to collaborate with one another not only across an individual workplace, but across the globe -- something a traditional whiteboard is incapable of achieving. Calico was created to allow any number of clients from various locations to interact on the same design space.

Extensibility allows the design process to change course over several iterations while still being able to document and track these changes. Flexibility is essential for a designer. The ability to rapidly add new features makes experimenting with new ideas relatively simple. Calico was created with this requirement of malleability in mind. Users need to be able to easily add plugins and new features.
\todo{maybe expand on plugins?} 
The requirement in the design of Calico to include extensibility allows for easy manipulation to the design and does not bind the designer to the tedious process of formally documenting their design.

Finally, reliability is essential for any tool so that the designer is not impeded by constant interruptions or crashes. 
\todo{we wanted this to be able to be run on a project indefinitely, so crashes were bad} 
The system must maintain the ability to stay online during the entire design process. Whiteboards do not ``crash'', and Calico needed to replicate that stability. 
\todo{expand on stability}
By using a centralized server, the system would have a significant increase in stability. Reliability is necessary for designers to be able to continuously use the system.

% One of the benefits of having a digital design interface was that the user could easily revert to any of their previous designs at the touch of a button.

Calico was created so that designers could freely expand on ideas through the use of a whiteboard. Designers can collaborate and make revisions to existing designs simultaneously. The power and flexibility offered by Calico allows users to more efficiently bring their designs to fruition. By using a whiteboard and projector, this sketching tool replaces the traditional whiteboard. By including a distributed system designers can work from any location. Extensibility is essential for a designer to add new features whenever necessary. The reliability of the system must be maintained throughout the design process. Calico combines the flexibility of a whiteboard with the technology of modern software to support designers during the design process.


% Is there another word we can use besides sketching? 
% Dont use WE

%%%%%%%%%%%%%%%%%%%%%%%%%%%%%%%%%%%%%%%%%%%%%%%%%%%%%%%%%%%%%%%%%%%%%%%%%%%%%%%%%%%%%%%%%%%%%%%%%%%%%%%%


% [Sketching in software design]
% [what limits current systems from doing what we do]
% [with respect to the needs we have, they fall short]
% [how does calico overcome those limitations]

% Formal design notations are best used for documenting a system, and do not work well when designing. This is the very reason that many designers turn to a whiteboard in order to sketch their designs. A platform that allows for sketching allows the designer to work without any restrictions at all [ON WHAT?] on designers and the drawings they are able to easily create. Designers are then able to freely manipulate their drawings without the burden of a structured design document. The software is able to aid the designer in creation of their drawing, but does not prevent the flexibility that a traditional whiteboard provides.  

% When working at the whiteboard, developers typically draw very informal table-napkin type drawings that are used to quickly build a solution to a given problem. Whiteboards are excellent tools for this, as they provide very little resistance to quick sketching. 

% The affinity that designers have for using a whiteboard has been recognized across many different disciplines of design. Sketching plays a very crucial, and very universal role in all applications of design. [this seems so weak?]

% Current software design tools offer the user a great amount of power when it comes to tool support. Designers can generate diagrams and other documentation artifacts using these tools. However, these tools do not seem to exist when it comes to sketching. Designers need a tool that is flexible and fluid like a whiteboard, but powerful enough to be useful to designers as a tool.  

% Calico is a software-sketching tool used by designers to help them easily draft potential software systems. Calico was designed to be used with a multi-touch whiteboard and projector so that it can replicate the role of a traditional whiteboard. We hoped to build upon the ease-of-use that a whiteboard provides designers, but while at the same time providing options and benefits that a traditional whiteboard system lacks. [I feel like I say this over and over] We hope that Calico can be used to be overcome the limitations of a traditional whiteboard, while not suffering the disadvantages of a structured design tool. While using a whiteboard is very fluid and unrestricted, a whiteboard does not provide any help to a designer -- it is identical to pen-and-paper drawings. 

% Our research has been about designing Calico, which is very similar to a traditional whiteboard in that it allows the user to sketch freely, but does not suffer from all of the disadvantages that a whiteboard does. One advantage Calico has over a traditional whiteboard is collaboration. Being a distributed, networked design system, Calico can enable designers to collaborate with one another across the globe – something that a traditional whiteboard could never achieve. [what do you think is the best ``story'' to tell here?]

% [RANDOM PARA]To improve the existing version of Calico, we decided to vastly improve the architecture in order to natively support collaboration. We had a few requirements that we wanted to satisfy with the new architecture. The first was the need for the system to be distributed. Originally, we wanted to support two boards that were placed next to each other and just provide a link between those two boards. However, in the end we decided against that and chose to create a system that would allow any number of clients, from various locations to interact on the same space. The second requirement was extensibility. We wanted to easily be able to add plugins and new features to Calico. The previous version proved to be very difficult to add new features, so our goal was to solve that problem, and easily allow new things to be added. The final requirement was reliability. We needed to have a system that was able to stay online, and retain the drawings that we created. Old versions of calico would crash, and then all designs would be lost. By having a centralized server, we hope that the server would be able to outlive client crashes and problems, and could be much more stable because it was not handling any graphical interface. With our new architecture we were able to continually experiment with new features. We could see which features were working well, and which ones were rarely used and could then be removed. The relative ease to adding new features made experimenting with new ideas very easy.


\chapter{Background}

To better understand the original reasoning for creating Calico, it is useful to have some insight into the history of sketching, and in particular the history of sketching in software design. 

\section{Sketching in Design}

Sketching has proven itself to have a very important role within the design community. 
Designers rely on sketching as part of their own thought process \cite{petre}. Sketching allows the designer a path by which they can go from their original thoughts to a more concrete design. Sketching acts as a very powerful tool that provides designers with fluidity as well as flexibility when translating ideas to drawings \cite{csik}. This flexibility and fluidity allows designers to focus on the ideas and discussion instead of focusing on \emph{how} the design is being recorded. There are many benefits to sketching. In the paragraphs below, we list the benefits that are most important.

First, sketching allows for a tight cycle of drawing, understanding, and reinterpretation. This cycle can be described as having a ``reflective conversation'' with the material \cite{schon}. This reflective conversation can lead to many ``unexpected discoveries'' \cite{suwa} that designers may not have noticed without sketching. Sketching allows designers to rapidly formulate new ideas, combine, transform, and also reject ideas

Second, sketching allows designers to quickly externalize their thoughts. This externalization allows for ideas to be viewed, analyzed, and discussed \cite{petre}. Designers may jot down ideas before they forget, or they may create a partial diagram to use as reference when explaining a concept to another designer. These sketches are known as ``thinking sketches'' for their ability to support the thinking process, as well as ``talking sketches'' because of their ability to support the discussion of an idea \cite{ferguson}. 

Third, sketching allows designers to use symbols to create representations of concepts. Once a concept has been reduced to symbols, designers can begin using spacial metaphors such as grouping or scaling to develop new insights \cite{larkin}. This also allows designers to create new notations if no suitable notation exists. Designers use different symbols for different activities, as well as different symbols for different phases within the same activity \cite{goel}. 

\section{Sketching in Software Design}

There have been many studies conducted that have indicated that sketching is used extensively in the field of software design \cite{petre, zannier, dekel}. One common use for sketching has been with teams of designers. Design teams utilize sketching and manipulate sketches to support their design conversations. Without being able to draw, it is difficult to explain ideas to others \cite{dekel, petre}.

Sketching has proven itself to be a valuable tool in the field of design. Studies have shown that sketching enables designers to share thoughts, ground ideas, manipulate concepts, and brainstorm \cite{cherubini}. Diagrams are essential tools for explaining concepts -- both in distributed and co-located teams \cite{yatani}.

In the context of software design, sketching tools prove to be even more useful. Zannier found that tools that encouraged conversations between designers gave way to decisions that considered more alternatives \cite{zannier}. Software design tools that do not force a specific design structure on the user can encourage a broader consideration of alternative designs, and can greatly improve the eventual design outcome \cite{zannier}. These tools allow designers to create several varying solutions in parallel, and then choose the best of these designs to continue designing. 

Software designers often sketch as a natural extension of the thought process used during the design phase to view more than a single solution simultaneously \cite{petre}. Sketching allows designers to have flexibility while they explore design problems. Sketching allows software designers to fluidly move focus between various potential ideas in order to contrast designs with others. It allows them to go from abstract thoughts to concrete ideas. Petre noted that sketching allows designers to shift focus among concepts in order to easily compare them with one another \cite{petre}. Petre also noticed that younger design sketches tend to be lacking in detail, but as the design grows and matures, more detail is typically added to the design.


\chapter{Calico}

% [anticpated/actual features]
% [purpose, how does it work]
% [supports designers in sketching]

\section{Canvas and Grid}

\begin{figure}[htb]
\centering
\includegraphics[width=0.8\textwidth]{grid.jpg}
\caption{The grid view within Calico}
\label{fig:grid}
\end{figure}
The grid is the focal point of any session in Calico.
It shows the various canvases that users may interact with in a given session.
Users may perform various operations on canvases from the grid, such as duplicating or clearing individual cells.
The grid also gives a clear overview of the designs that are happening in a session.

\section{Gestures}

\section{Scraps}
Scraps in Calico can be thought of as ``scraps of paper'' that one would place on a desk or on a white board.
Scraps can be easily relocated to different parts of the screen, or even other canvases.
Scraps can be stacked on top of each other and then treated as a unit or group.
By treating scraps as if they were pieces of paper, we [make it easy to understand the manipulation], as designers can easily relate Calico to their current design 

\section{Palette}
The palette in Calico provides users with a ``drawer'' that can easily be used to store commonly used shapes and artifacts.
The palette can be synchronized across sessions so that other users in the session can share the same palette.

\chapter{Objectives}

% PARA
% when wefirst examined calico, there were a number of problems
% LIST SOME PROBLEMS
% this caused us to take a step back an dlook at the current arch, current design, and current set of requirements and take a holistic view of the project
When we first examined early versions of Calico, there were a number of problems that existed. 
Some of the problems that existed were: lack of native support for distributed users,
collaboration between designers was not natively supported,
lack of a generic plugin interface to add new features, 
and sluggish input handling. 
In addition to these problems, there were also a number of additions that we felt would be beneficial to future users of Calico.
These shortcomings caused us to take a step back and reexamine the current architecture, current design, and current set of requirements that we had for the original versions of Calico.


% when rethinking our architecture, we came up with the following objectives
% - must be natively distributed ::: original calico was built and then collaboration was added on later which did not work. we wanted to design an arch that was distributed from the start
% END
When rethinking the architecture of Calico, we decided on the following objectives for the new version
\begin{itemize}\itemsep1pt
\item 
\textbf{Must natively support distributed designers.}
\newline
The original version of Calico was created as a single-user application that was not distributed at all. In a later version of Calico networking support was added as a ``patch,'' but was not natively supported, and because of this the overall performance suffered.
We wanted to design an architecture for Calico that was distributed from the start.

\item
\textbf{Must natively support collaboration between designers.}
\newline
As described in the previous paragraph, the original version of Calico was not collaborative. 
Collaboration was added later as a hack, but it was not able to properly deal with conflicts between designers modifying the same object at the same time. 
Given that the new version of Calico was to be distributed from the start, we needed to ensure that collaboration was also natively supported.
Designers needed to be able to work in parallel within the same workspace, and conflict resolution was a clear objective in the new version of Calico. 

\item
\textbf{Must provide a generic interface for external plugins.}
\newline
The original version of Calico provided no interfaces for external plugins to interact with. 
Adding new features to the original Calico was always done as ``hack'' which always created more problems.
Due to the fact that plugins were being created as patches to the core, there was no way to easily disable or enable specific features - users were forced to run with the full feature-set. 
With the next iteration of Calico, we wanted to prevent this by providing an easy-to-use plugin system that would allow developers to create third-party plugins that could be easily added to Calico, and could be enabled or disabled on demand.

\end{itemize}


In addition to solving issues that we noticed in Calico, we also wanted to use this opportunity to optimize existing design choices that were used.
We identified a number of features that we decided should be implemented in the new version of Calico.
While these features were not fixing critical bugs, we hoped that these additions would ensure that Calico would be a viable product in the future. 
The features that we decided to add were:
\begin{itemize}\itemsep1pt
\item 
\textbf{Should provide an easy-to-use administration interface.}
\newline
Calico was on it's way to being used in many different places, and we needed an easy way to manage the servers. 
The previous version of Calico was a standalone peer-to-peer service, so there was no administration interface provided to the user.
We decided that the Calico server should also provide a web-based interface that would be easy for users to interact with, and would allow users to perform all the necessary administrative functions.

\item 
\textbf{Network usage should be optimized to reduce congestion}
\newline
The previous version of Calico required a very fast connection when working with other users. If a user was on a slower connection, then they would see updates well after they were made, and this delay made Calico almost unusable in any location other than local Ethernet.
We wanted to reduce the bandwidth that was required to use Calico, in the hopes that designers would be able to collaborate remotely over the Internet -- even in other countries.

\item 
\textbf{Input handling response should be improved.}
\newline
One goal of Calico is to replace a traditional whiteboard with a digital version. 
In order to accomplish this, we needed to reduce the input-response delay as much as we could. 
If a user were to quickly scribble some text on the canvas, Calico had to be able to keep up without lagging. 
Previous versions of Calico had a bloated event handling architecture which meant that faster a designer drew on the screen, the slower the response was. This lag was not acceptable, and we hoped that an overhaul of the input handling system would allow Calico to keep pace with even the quickest designers.

\end{itemize}


% obj#2, collaborative from the start. original calico was not collaborative. conflicts were not being dealt with.
% in the new architecture, given that we are going to be distributed, we must ensure from the start that users are able to collaborate

% (one-liner, then explanation of the problem)

% PROBLEMS
% must be natively distributed
% must support collaboration & users working in parallel
% add administrative interface - calico was on its way to be used in many different places, we needed a way to manage all the servers. previous version was just standalone and had no administration. want a nice way to manage/start/stop/etc the server
% bloated networking system
% no generic plugin system for adding new features
% sluggish input handling

%%%%%%%%%%%%%%%%%%%%%%%%%% OLD %%%%%%%%%%%%%%

%[old version was single user, list limitations] 
% Calico was originally designed to be a single-user system that allowed individuals to operate an electronic whiteboard. While this original design worked great in an isolated environment, it made it very difficult for users to collaborate with one another. Users had to resort to taking screenshots and emailing them to one another.  Our aim with the new version of Calico was to create a system that would function in an isolated environment, but could easily facilitate collaboration with multiple users when needed. By designing with collaboration in mind, we were able to create a client-server architecture that would support many users interacting with the same canvas area simultaneously. The server had to be responsible for maintaining order of objects, as well as acting as a version control to make sure that users were not able to perform conflicting actions that would corrupt the drawings.

% As an experiment, we worked to add collaboration to the original Calico implementation. This was able to help us determine that using Calico in a multi-user environment would be much more useful than we originally expected, and drove us to create a more robust multi-user implementation. Our attempts to make old calico multi-user were not successful, as the system was extremely sluggish and frustrated many users. We found that our users expected the system to be much more responsive, and the sluggishness was not acceptable. This was one of the biggest reasons that we decided to redesign a new Calico from the ground up.  

% \todo{reasoning for new iteration, rethinking architecture}
% Rather than continue working on the existing version of Calico, we discussed the usefulness of redesigning the architecture from the ground up in order to better support the features we were hoping to add. We wanted to create a system based on the client-server architecture that would facilitate many clients all interacting with the system at the same time. The old version of Calico used a peer-to-peer connection that limited the number of active clients that could use the system. The old peer-to-peer architecture proved to be highly problematic when connecting with users over the Internet. While the system was able to work fine locally, the ultimate goal for the system was to collaborate with users all over the world, and we realized that a peer-to-peer architecture would not be ideal when collaborating over the Internet. This was one of the reasons we chose to move to a client-server architecture. We were able to place the server on a system that was open to the Internet, and it would eliminate all the connection problems that plagued the old peer-to-peer architecture. 

% Another area where the client-server architecture benefitted us was with performance. We created a server that was highly optimized for processing data from many different clients at the same time. This allowed us to handle many more clients without any noticeable drop in processing time. Previous versions of Calico had the clients act as servers, which meant that in addition to processing all of the graphical data, they were also responsible for processing all incoming data and drawing what the other clients were sending.
% %[list requirements and wants/needs]

% %[req:persistence]
% The first iteration of Calico was a peer-to-peer system. This meant that there was no single place where the drawings would be stored. There was no way to view the contents of the session unless you were actively viewing it from within the Calico program itself. We wanted to be able to access the current state of a session without the need to actually join the session itself. By creating a central server that was responsible for maintaining the session, we were able to have a persistent history of the session. Users could join and disconnect, and then return, and still be able to undo operations that were performed long ago. Users could import canvas drawings into other programs by requesting a rendered image from the server. By storing the session state at a single location, we reduced the likelihood that data would become corrupted in transit, or data that would be corrupted synchronizing between many ``master'' servers as it had in the traditional peer-to-peer architecture. The persistent server was regarded as the true master, and if any client differed, it would synchronize with the central server, rather than assuming its own state was the ``correct'' version.

% %[req:sessions]
% Sessions was not initially planned in our rewrite of Calico, but we soon found a need for sessions to be added. The previous version of Calico had no session system at all, however it was not really necessary because users could essentially create their own sessions by just starting another instance of the program. This provided an easy method for users to work on various projects without interfering with the designs of another project. With the new client-server architecture, it was more difficult to start a new instance of the server in order to work on a separate project. Thus the need for sessions was realized, as designers needed a way to easily create a separate Calico instance that would not interfere with the existing session. 

% %[req:admin interface]
% One of the benefits of having a central server was the ability to have an ``administrative interface'' that could allow users to perform actions on the server, and backup/restore sessions. With the new version of Calico, we opted to create a web-based administrative interface that let the user perform various low-level commands that were used for debugging. Along with the ability to perform commands, users could download a file containing the entire state of a session. These files could then be restored at a later point in time, and would restore the session to its previous state. This proved to be very useful during the initial development, as it provided a safety net for users. Users were able to experiment more knowing they had a backup of their designs.


\chapter{High-Level Architecture}
% high level architecture
% how accomplish objectives from above
% what changes, rationale
% how it actually works
% checksums/persistence/how model is updated
% dealing with parallel edits
% updating clients
% presence awareness
% 1) architecture  (why) [distributed/reliable/multiple boards]
% 2) "new htink" - why
% design decisions
% new archi
% experience with it

\todo{Need a diagram to show how stuff connects}

In order to meet the objectives that we set in the new version of Calico, we decided that the architecture needed to be completely redesigned in order to support all the changes. The new architecture that we chose had to be able to support all the features, as well as easily support multiple users simultaneously. It also had to be accessible from anywhere, and had to be stable to prevent constant headaches when recoverying from crashes.

%\begin{figure}[h]
%\centering
%\includegraphics[width=0.8\textwidth]{calico_arch.png}
%\caption{Overview of Calico's Architecture}
%\label{fig:calico_arch}
%\end{figure}

The first change made was a switch from the original peer-to-peer system to a completely new client-server architecture. We believed that by using the client-server architecture, we could provide a centrally accessible service that would easily support multiple clients and still maintain stability. The server could act as a headless service that could be very powerful since it did not have to perform any client functions. This meant that the server could be dedicated to the task of managing all the interactions between clients, and could be much more stable than in a peer-to-peer context.

% Major change - java objects no longer used
One of the significant changes from the original Calico was the change in network communication design. The previous version of Calico had networking that was designed as a plugin, and was not fully integrated into the system. In this old design, when a change was made by another client, the entire change was saved to a Java object. This object was then serialized and sent in full over the network. The receiving client would deserialize this data, and then perform the change on its local drawing canvas. This process was very easy to implement, but had the cost of enormous overhead due to the constant serialization/deserialization process that would take place for each action. It also incurred the overhead of Java serialization, which would add much more data to the final packet then would ever be needed. All this overhead would quickly become apparent to users when the system would lag in response to many actions done in quick succession. Our goal was to streamline this process as much as possible so that we could reduce lag when the system was being heavily utilized.

To improve the responsiveness of Calico, and reduce the network overhead, I created a custom packet design that could be used by Calico to notify both the client and server when various actions were performed. These special packets provided exactly what was needed to effectively communicate changes to clients, and were reduced to the minimum size. Packets were byte arrays that were written directly to the wire, and based on various settings, they could be decoded back into their specific components. 



\begin{figure}[h!]
\centering

\begin{bytefield}{128}
  \bitbox{32}{packet size} & \wordbox{32}{command ID} & \bitbox{64}{command-specific data}
\end{bytefield}

\caption{Calico Packet}
\label{fig:calico_packet}
\end{figure}

This packet design was loosely based on the packet design of the Half-Life game engine \cite{todo}.As seen in the figure above, the packet was broken up into three main parts. The first four bytes of the packet was the length of the packet. This would tell the network system how much farther it needed to read to ensure it received the entire packet. The next four bytes held the command identifier. This number was used to tell the network system what this packet was about. Both the client and the server had a list of commands and their IDs, which was used to translate from a programmer-friendly command name to the command ID number. The remaining bytes could vary based on the specific command that was given. Each command had different parameters, which both the client and server were aware of, and could read the packet. By writing data directly to the wire, there was no overhead at all with inflated packet sizes -- each packet was as small as it could be. This made the network communication between the client and server incredibly efficient, and helped to reduce the lag, even when put under heavy usage.


%\section{Major Changes}
% move away from java objects being sent over the wire
% describe packet system

%\section{Maintaining Consistency}
% How data is sent back and forth between server and client
% How data is written to the controllers
% how parallel edits are dealt with
% 

%\section{Server Architecture}


%\section{Client Architecture}

% Components
% - Server
% - Client

% Implementation (broken into sections)
\chapter{Implementation}

% cleint handling
% server handling
% database/storage
% gesture recognition

% talk about actual implementation here. discuss controllers, how entities are stored.

% key data structures
% how are things stored
% what modules are available to the client and server
% where is the plugin architecture
% think of a picture

% how you write plugins. maybe show an example plugin (or snippet)

% "how do things work"

% 5 = architecture & implementation 
% 5.1 = key architectural design decisions
% 5.2 = implementation OR components


% - diagrams
% - expansion 
% -- plugin example (skeleton)
% - talk about input handling system (how i know what object you touch)
% - admin web server
% - "if this is what the screen looks like, then how are all the things stored"

\section{Networking}

\begin{figure}[htb]
  \centering
  \includegraphics[width=0.8\textwidth]{network.png}
  \caption{Overview of Network Flow in Calico}
  \label{fig:network}
\end{figure}
% Intro parts
As a distributed and collaborative system, networking is a key part of Calico. The networking system must be able to support many clients at the same time, many of who may be working on the same workspace. The networking system in Calico consists of three major components: The main server socket and individual client process threads, the network queue processor, and the \texttt{CalicoPacket}. These three components work together to ensure that client requests are quickly handled, and that conflicts are properly managed.

% Socket/ClientThreads
The first component is the primary server socket and client threads. We decided to implement Calico as a TCP\cite{network} server in order to benefit from TCP's flow control and error correction. While TCP may be slower than UDP, the benefits of ensuring packets are received in the order they are sent as well as maintaining a continuous session outweighted any performance gains of UDP. When the Calico server is started, a TCP socket listener is created that waits for client connections to arrive. Upon receiving a new connection from a client, a new \texttt{ClientThread} is created that will then handle the connection with the client for the remainder of the client's session. By using a multi-threaded network system, we can easily maintain multiple sessions at the same time.

% ProcessQueue
The second component is the network queue processor, implemented as \texttt{ProcessQueue} class. This class is responsible for processing each packet that arrives from a client. When one of the \texttt{ClientThread}s receives a packet, the packet is then routed to the \texttt{ProcessQueue} which has a method for each packet type that can process that specific command. This class can then call any controllers as needed in order to complete the requested command.

% CalicoPacket
The \texttt{CalicoPacket} is responsible for encoding all information of a specific command into a byte array that can then be sent to clients. The packet is the lowest level of the networking system, but one of the most important. As noted in the previous section, the Calico packet was heavily based on Half-Life's network packet design. The first part of the packet contains a four-byte length which tells the system how far it needs to read in order to obtain the entire packet. By breaking commands down and transmitting them at the byte level, we were able to greatly reduce the network overhead that would be created if we had decided to send serialized objects over the network instead.


\section{Object Storage}
In Calico, all core components are stored using high performance Java hash maps. 
We decided that using FastUtil's\cite{fastutil} high performance maps would provide a better response time as opposed to using Java's core HashMap class. 
Each controller contains a single hash map that acts as a database for all objects that the controller is responsible for.
All access to hash maps requires use of a ``key'' to perform a lookup.
All objects are given a sequential 64-bit integer that acts as a globally unique identifier for each object.

One important design decision that was made was the choice to use a sequential identifier for each new object. The two factors that influenced our decision were: the relatively small size of a 64-bit integer in comparison with a universally unique string (which would be almost double the size), and better client performance.
Small size was very important to us - this identifier would be included in nearly every network packet that was sent to and from the server. We wanted the footprint to be as small as possible, but we needed something that we could ensure would be unique for every object. 
The second decision - better client performance - was another benefit provided by using a sequential identifier. Clients could request a pre-allocated block of identifiers that they could assign to objects created client-side. The client could then create objects and assign them an identifier and submit the object directly to the server. Due to the fact that identifiers were pre-allocated, the client did not have to wait for confirmation from the server in order to add the object to its local hash map.

\begin{figure}[htb]
  \centering
  \includegraphics[width=0.8\textwidth]{scraps.png}
  \label{fig:scraps_storage}
\end{figure}

The screenshot shown above displays a typical design session in Calico. The content shown above would be stored in the various databases within each controller. This image represents a single \texttt{Canvas} object which would be stored within the \texttt{CanvasController} class. The \texttt{Canvas} object would have a list that contains the identifiers for each object that is being displayed on screen (five \texttt{Scrap}s, two \texttt{Arrow}s, and many \texttt{Stroke}s). Each of these objects would be stored in their respective controller. Each of these objects would require a list of \texttt{Point}s that would form the boundaries of the object. These are used within Calico to determine the location of an object, as well as whether an object is ``contained'' within another object (such as a scrap). Each object can have additional attributes that are specific to its type (arrows would need endpoints, strokes would have a color). These attributes would be stored within the object itself and would be converted to a \texttt{CalicoPacket} whenever a new client connected and needed to load the existing state of Calico.


\section{Object Controllers}
In Calico the object controllers are responsible for handling all operations made on elements under their control. Each object type in Calico was provided with a controller.
Scraps, strokes, arrows, and canvases each have a controller that handles any interaction performed on the object. 
Controllers also act as the storage points for objects they are responsible for. This storage is discussed in the next section.

Controllers were created as static classes. This meant that each method in the controller would always be provided with the object's identifier, which it could then use to locate the object within the database. 
The controllers were also responsible for notifying the server of any changes that were performed. However, in order to prevent clients from sending notifications for actions that were not performed by the user, we created two types of functions for each action in the controller. The first function would perform the action and then notify the server, while a second function would perform the action without sending notifications. Any operation that was performed by the user would have a notification sent to the server. Any operation that was the result of receiving input from the server would not send a notification.

\section{Plugin Framework}
To improve the extensibility of Calico, a plugin framework was created. This plugin framework allowed for extra features to be easily integrated into Calico. Plugins could subscribe to specific events that they wanted to be notified about. When one of these events was triggered within Calico, all subscribers to that event were notified, and could perform any action based upon the event information. 

\begin{figure}[htb]
  \centering
  \small
  \verbatiminput{figures/plugin.java}
  \normalsize
  \caption{Example plugin source code}
  \label{fig:plugin_file}
\end{figure}

To create a plugin, developers only had to extend a provided abstract plugin class that allowed each plugin to register itself with a \texttt{PluginManager} that was responsible for publishing network events and interface events to each plugin. Plugins were provided with default ``hooks'' that would be called upon when the plugin was loaded or unloaded. Plugins were able to call a method \texttt{registerNetworkCommandEvents} that would allow it to subscribe to specific network commands that it was interested in. Another method, \texttt{RegisterPluginEvent} enabled the plugin to subscribe to various interface events that could be used for processing user input and clicks. This plugin system provided a very easy way for developers to interact with the Calico framework, without needing to modify the code of the existing system. Developers could quickly create plugins that could interact with a live system.




\section{Input Handling}
% - talk about input handling system (how i know what object you touch)
% when the mouse pressed, locate all scraps that contain the mouse location
% determine the smallest scrap that contains the location
% actions are performed on that
% input handler is then "locked" onto that specific group (so that during the move, we are not trying to calculate over and over)
% unlocked as soon as it is released
% each object knows the list of points (or path) that acts as its bounds.
% Each object can then know if a specific point is within
In an effort to improve the user interaction experience, we felt that the existing input handling system needed to be created specifically for Calico. The previous version of Calico had relied on the input handler provided by the drawing framework that was used. This became unusable when working with canvases containing hundreds of objects. To improve response times, we need to recreate a newer input handler that did not rely on the drawing framework - meaning that it could continue to operate even while the drawing framework was busy rendering the display.

Rather than having a specific mouse listener linked to each object on screen, we instead decided to have a global mouse listener that could determine which object was being touched. This meant that mouse input only needed to be processed by a single handler, and based on various modes, it could intelligently handle interaction with the user. 

Each canvas maintained a list of all objects that were present on screen. Each object also was responsible for maintaining a list of coordinates that formed the ``bounds'' of that object. Objects could then easily determine if a specific point was contained with the ``bounds'' of that element.
Upon receiving any mouse input, the input handler would then iterate through the elements and determine which elements contained the mouse location. This list would then be further reduced to return only the \emph{smallest} object that contained the mouse location. This was very helpful when handling scraps that had been stacked on top of each other - the parent scrap still contains the mouse location, but clearly the user wants to interact with one of the child elements.

After locating the element that the user will be performing actions with, the input handler would then be ``locked'' to only interact with that element until the mouse has been released. This meant that users could move a scrap around screen (and move the mouse outside the bounds of the selected scrap) and still have the scrap follow the mouse location. Another benefit of locking the input handler on a specific element was that while doing very intensive operations (such as moving an element across the screen) we were not wasting resources to recalculate which element we were interacting with. This is something that was not availble in the previous versions of Calico.

The one drawback to using this approach meant that the object location and screen coordinate systems were identical. What that meant was that the Calico coordinate system was equal to the user's screen resolution. If a user with a larger screen joined the session, they could draw shapes and figures that were outside the viewable area of users with smaller screens. At the time, this was not a problem because all users were running the same resolution, so fortunately this problem rarely occurred.

\section{Administrative Interface}
% - admin web server
% uses httprequesthandlers
% velocity templates
% based off phpBB admin interface
% has a few basic functions, and allows admins to administer those
% view connected clients
% update configuration values
% upload images
% execute specific commands
% perform backup
% restore backups
One of the last components to be added was the administrative interface. Calico now had a headless server, and we needed a way to remotely manage it. We decided against building the interface within the Calico client - we wanted to keep the client as lightweight as possible. 
We decided to create an easy to use web-based interface that would allow the Calico server to be managed easily using a web browser.

To keep the admin system lightweight, we decided to just utilize simple \texttt{HttpRequestHandler}s that were tied to a single \texttt{HttpService} listening for web connections. All frontend HTML and CSS was generated using Velocity\cite{velocity} templates. The interface look-and-feel was copied from the administrative interface of an open-source forum system known as PhpBB\cite{todo}.

The administrative interface only provided basic management abilities. The core abilities it provided were:
\begin{itemize}\itemsep1pt

\item
\textbf{Easily modify settings}.
We needed the ability to modify the settings of an existing Calico instance easily. What was created was a page that listed all the configuration variables and easily let the administrator modify those values. When saved, the Calico server could then be restarted if needed to reflect the configuration changes.

\item
\textbf{View currently connected clients}. 
This was helpful in the classroom environment to be able to see all users who were currently connected to the server. In the future we planned to add the ability to ``kick'' clients from the server, but this was never full developed.

\item
\textbf{Perform backups and restoration of sessions}.
One of the major requirements was the ability to easily generate a backup of an entire session. Our goal was to provide an easily way to download a backup file that represented the current state of the server. It needed to contain all the data required to fully restore all canvases and any scraps, strokes, or arrows that may be contained in the canvas. Once we created the backup system, we needed a way for a server to restore any backup. Our administrative interface provided an easy to use form where a backup file could be selected and then uploaded directly to a running server. The server would then read the contents of this file and import all data into the server. Restoring a backup essentially wiped any existing content and recreated the entire session from scratch. Once a restoration had been performed, clients could reconnect and they would see all the data, just as it was when the backup was created.

\end{itemize}

These few requirements were the driving force behind the administrative interface. It stands as a very barebones and minimalistic interface that provides functions that we decided were essential to operation. This administrative interface has proved invaluable when generating and restoring backups. Having a backup of sessions provides users with a greater level of reassurance that any work they have made will not be lost should anything happen to the server.



\section{Networking}

\begin{figure}[htb]
  \centering
  \includegraphics[width=0.8\textwidth]{network.png}
  \caption{Overview of Network Flow in Calico}
  \label{fig:network}
\end{figure}
% Intro parts
As a distributed and collaborative system, networking is a key part of Calico. The networking system must be able to support many clients at the same time, many of who may be working on the same workspace. The networking system in Calico consists of three major components: The main server socket and individual client process threads, the network queue processor, and the \texttt{CalicoPacket}. These three components work together to ensure that client requests are quickly handled, and that conflicts are properly managed.

% Socket/ClientThreads
The first component is the primary server socket and client threads. We decided to implement Calico as a TCP\cite{network} server in order to benefit from TCP's flow control and error correction. While TCP may be slower than UDP, the benefits of ensuring packets are received in the order they are sent as well as maintaining a continuous session outweighted any performance gains of UDP. When the Calico server is started, a TCP socket listener is created that waits for client connections to arrive. Upon receiving a new connection from a client, a new \texttt{ClientThread} is created that will then handle the connection with the client for the remainder of the client's session. By using a multi-threaded network system, we can easily maintain multiple sessions at the same time.

% ProcessQueue
The second component is the network queue processor, implemented as \texttt{ProcessQueue} class. This class is responsible for processing each packet that arrives from a client. When one of the \texttt{ClientThread}s receives a packet, the packet is then routed to the \texttt{ProcessQueue} which has a method for each packet type that can process that specific command. This class can then call any controllers as needed in order to complete the requested command.

% CalicoPacket
The \texttt{CalicoPacket} is responsible for encoding all information of a specific command into a byte array that can then be sent to clients. The packet is the lowest level of the networking system, but one of the most important. As noted in the previous section, the Calico packet was heavily based on Half-Life's network packet design. The first part of the packet contains a four-byte length which tells the system how far it needs to read in order to obtain the entire packet. By breaking commands down and transmitting them at the byte level, we were able to greatly reduce the network overhead that would be created if we had decided to send serialized objects over the network instead.
\section{Object Storage}
In Calico, all core components are stored using high performance Java hash maps. 
We decided that using FastUtil's\cite{fastutil} high performance maps would provide a better response time as opposed to using Java's core HashMap class. 
Each controller contains a single hash map that acts as a database for all objects that the controller is responsible for.
All access to hash maps requires use of a ``key'' to perform a lookup.
All objects are given a sequential 64-bit integer that acts as a globally unique identifier for each object.

One important design decision that was made was the choice to use a sequential identifier for each new object. The two factors that influenced our decision were: the relatively small size of a 64-bit integer in comparison with a universally unique string (which would be almost double the size), and better client performance.
Small size was very important to us - this identifier would be included in nearly every network packet that was sent to and from the server. We wanted the footprint to be as small as possible, but we needed something that we could ensure would be unique for every object. 
The second decision - better client performance - was another benefit provided by using a sequential identifier. Clients could request a pre-allocated block of identifiers that they could assign to objects created client-side. The client could then create objects and assign them an identifier and submit the object directly to the server. Due to the fact that identifiers were pre-allocated, the client did not have to wait for confirmation from the server in order to add the object to its local hash map.


\begin{figure}[h!]
  \centering
  \begin{subfigure}[t]{0.4\textwidth}
    \centering
    \small
    \verbatiminput{figures/java/storage_stroke.java}
    \normalsize
    \caption{Stroke}
    \label{code:stroke_storage}
  \end{subfigure}%
  ~ %add desired spacing between images, e. g. ~, \quad, \qquad etc. 
    %(or a blank line to force the subfigure onto a new line)
  \begin{subfigure}[t]{0.4\textwidth}
    \centering
    \small
    \verbatiminput{figures/java/storage_scrap.java}
    \normalsize
    \caption{Scrap}
    \label{code:scrap_storage}
  \end{subfigure}
  \caption{Calico object representations}
  \label{code:storage}
\end{figure}

Figure \ref{code:storage} shown above shows snippets from two objects within Calico. The first is the \texttt{Stroke} element (which provides lines) and the \texttt{Scrap} object. A graphical depiction of scraps can be seen in figure \ref{fig:scraps_storage} as the light gray areas. Scraps can be parented within other scraps (if a scrap is completely contained within another scrap, then the larger scrap would be considered the ``parent'' of the smaller scrap).

Figure \ref{code:storage} shows how all elements within Calico are provided with a unique 64-bit \texttt{uuid} which acts as the ID for the object. The second field, \texttt{parentUUID} is the ID value of the parent object that contains the current object. If the current object is not parented, then this would be set to \texttt{0L}. The third field is the \texttt{canvasUUID}. This value would always be present, as it links the current element to the specific canvas that it was created on. Every element within Calico must be created on an existing canvas. 

The last field that is shared among Calico elements is the \texttt{points} field. We decided to use Java's native \texttt{Polygon} class to represent the coordinate path of the given element. By using a \texttt{Polygon} object, we were able to use native geometry operations to determine containment and intersection of elements. When a stroke is being drawn on a client machine, the server will continously append \texttt{Point2D} objects to the \texttt{points} field for the respective Stroke. 

The \texttt{Scrap} class shown in figure \ref{code:scrap_storage} also includes three additional sets: \texttt{childScraps}, \texttt{childStrokes}, \texttt{childArrows}. These sets contain a list of UUIDs that can be mapped to any element that is a ``child'' of this scrap. These sets are accessed whenever a parent scrap is deleted or moved, because all operations must be executed on all child elements.


\begin{figure}[h!]
  \centering
  \includegraphics[width=0.8\textwidth]{scraps.png}
  \caption{Calico design session}
  \label{fig:scraps_storage}
\end{figure}

The screenshot shown in figure \ref{fig:scraps_storage} displays a typical design session in Calico. The content shown above would be stored in the various databases within each controller. This image represents a single \texttt{Canvas} object which would be stored within the \texttt{CanvasController} class. The \texttt{Canvas} object would have a list that contains the identifiers for each object that is being displayed on screen (five \texttt{Scrap}s, two \texttt{Arrow}s, and many \texttt{Stroke}s). Each of these objects would be stored in their respective controller. Each of these objects would require a list of \texttt{Point}s that would form the boundaries of the object. These are used within Calico to determine the location of an object, as well as whether an object is ``contained'' within another object (such as a scrap). Each object can have additional attributes that are specific to its type (arrows would need endpoints, strokes would have a color). These attributes would be stored within the object itself and would be converted to a \texttt{CalicoPacket} whenever a new client connected and needed to load the existing state of Calico.





% \lstset{ %
%   language=Octave,                % the language of the code
%   basicstyle=\footnotesize,           % the size of the fonts that are used for the code
%   numbers=left,                   % where to put the line-numbers
%   numberstyle=\tiny\color{black},  % the style that is used for the line-numbers
%   stepnumber=2,                   % the step between two line-numbers. If it's 1, each line 
%                                   % will be numbered
%   numbersep=5pt,                  % how far the line-numbers are from the code
%   backgroundcolor=\color{white},      % choose the background color. You must add \usepackage{color}
%   showspaces=false,               % show spaces adding particular underscores
%   showstringspaces=false,         % underline spaces within strings
%   showtabs=false,                 % show tabs within strings adding particular underscores
%   frame=single,                   % adds a frame around the code
%   rulecolor=\color{black},        % if not set, the frame-color may be changed on line-breaks within not-black text (e.g. comments (green here))
%   tabsize=2,                      % sets default tabsize to 2 spaces
%   captionpos=b,                   % sets the caption-position to bottom
%   breaklines=true,                % sets automatic line breaking
%   breakatwhitespace=false,        % sets if automatic breaks should only happen at whitespace
%   title=\lstname,                   % show the filename of files included with \lstinputlisting;
%                                   % also try caption instead of title
%   % keywordstyle=\color{blue},          % keyword style
%   % commentstyle=\color{dkgreen},       % comment style
%   % stringstyle=\color{mauve},         % string literal style
%   escapeinside={\%*}{*)},            % if you want to add LaTeX within your code
%   morekeywords={*,...}               % if you want to add more keywords to the set
% }

% % \lstset{language=C}
% % \begin{lstlisting}
% % #include <stdio.h>

% % int main(int argc, char ** argv)
% % {
% %   printf("Hello world!\n");
% %   return 0;
% % }
% % \end{lstlisting}

% \lstinputlisting[language=Java]{figures/java/storage_stroke.java}
\section{Object Controllers}
In Calico the object controllers are responsible for handling all operations made on elements under their control. Each object type in Calico was provided with a controller.
Scraps, strokes, arrows, and canvases each have a controller that handles any interaction performed on the object. 
Controllers also act as the storage points for objects they are responsible for. This storage is discussed in the next section.

Controllers were created as static classes. This meant that each method in the controller would always be provided with the object's identifier, which it could then use to locate the object within the database. 
The controllers were also responsible for notifying the server of any changes that were performed. However, in order to prevent clients from sending notifications for actions that were not performed by the user, we created two types of functions for each action in the controller. The first function would perform the action and then notify the server, while a second function would perform the action without sending notifications. Any operation that was performed by the user would have a notification sent to the server. Any operation that was the result of receiving input from the server would not send a notification.

\begin{figure}[htb]
  \centering
  \small
  \verbatiminput{figures/java/controller.java}
  \normalsize
  \caption{Partial source of Calico's Stroke Controller}
  \label{code:controller}
\end{figure}
% class StrokeController {
%   public static void start(long uuid, long cuid, long puid, Color color);
%   public static void append(long uuid, int x, int y);
%   public static void finish(long uuid);
%   public static void delete(long uuid);
%   public static void move(long uuid, int deltaX, int deltaY);
% }

Figure \ref{code:controller} shown above gives you an excerpt of code from one of Calico's controllers. This example shows the basic operations that are available for the \texttt{StrokeController}. This controller is responsible for handling all ``stroke'' elements (line drawings). The controller was designed to be globally available backend that could be accessed by both the network connection (so that strokes could be created by the server) as well as by the input system (so strokes could be created using mouse events). We will imagine a typical scenario which we will use to describe how the Calico client would normally interact with the controller to create a sketched line on screen.

In our example, we start with a blank \texttt{Canvas} screen, ready to begin sketching. The first event would be a \texttt{MouseEvent.MOUSE\_PRESSED} event that would be sent to Calico's input handlers. The input handlers are responsible for ensuring that the user is in the proper mode (for sketching) and that the user is not clicking on a button. Once the input handler has determined that the user intends to sketch on screen, then the event is handed off to the controller.

We first need to extract the \texttt{x} and \texttt{y} positions of the mouse coordinates from the \texttt{MouseInputEvent} object. The next step would be to obtain a new \texttt{uuid} for our stroke. This requires a call to \texttt{Calico.uuid()} which provides us with a new 64-bit integer that is unique across all clients. We now have all the information needed to start our stroke element, so we will execute a call to \texttt{StrokeController.start(uuid, canvasUUID, 0L, Color.RED)}. This will create a new stroke that is colored red, and has the provided \texttt{uuid}. 

Our stroke has been started, but at this moment there are no coordinates attached to the stroke. We will next need to call \texttt{StrokeController.append(uuid, xPos, yPos)} using the \texttt{x} and \texttt{y} positions of the mouse that we extracted in the previous paragraph. The call to \texttt{append} will repeate for every \texttt{MouseEvent.MOUSE\_DRAGGED}, until the \texttt{MouseEvent.MOUSE\_RELEASED} event is fired.

When the mouse has been released, Calico will make a call to the \texttt{StrokeController.finish(uuid)} method which signals that the user has finished drawing the stroke element, and it should be finalized. During the \texttt{finish} method, we will ensure that the stroke is properly parented (if it is contained within a scrap, it will be added to that scrap).

During this process explained above, the client would be sending events to the server (\texttt{STROKE\_START}, \texttt{STROKE\_APPEND}, \texttt{STROKE\_FINISH}). When the server receives these events, then it will perform an identical set of controller calls to record the sketching server-side. It will also mirror these commands to all connected clients. The benefit to using a global controller on the client is that when a stroke (or any element) is created by another user, the \texttt{CalicoPacket} can be easily mapped to the proper controller methods.
\section{Plugin Framework}
To improve the extensibility of Calico, a plugin framework was created. This plugin framework allows for new features to be integrated into Calico without needing to directly modify its core code. Plugins subscribe to specific events that they wanted to be notified about. When one of these events is triggered by Calico, all subscribers to that event are notified, and can choose to perform any action based upon the event information. 

\begin{figure}[h!]
  \centering
  \small
  \verbatiminput{figures/plugin.java}
  \normalsize
  \caption{Example plugin source code}
  \label{code:plugin_file}
\end{figure}

To create a plugin, developers only have to extend a provided abstract plugin class that allows each plugin to register itself with a \texttt{PluginManager} that is responsible for publishing network events and interface events to each plugin (see Figure \ref{code:plugin_file}). Plugins were provided with default ``hooks'' that are called when the plugin is loaded or unloaded. Plugins call a method \texttt{registerNetworkCommandEvents} to subscribe to specific network commands that it was interested in. Another method, \texttt{RegisterPluginEvent}, enables the plugin to subscribe to various interface events that are used for processing user input and clicks. For instance, plug-ins can register to be notified when $\mathcal{X}$ or $\mathcal{Y}$ events happen.

\subsection*{Interaction with Core Elements}
We decided to allow plugins to access the core Calico controllers to be able to create, modify, and delete core elements (\texttt{Scrap}s, \texttt{Stroke}s, \texttt{Arrow}s). For example, we created a chat robot that would listen for incoming text messages, and upon receiving a message, creates a new \texttt{Scrap} with the given message as the contents. In another example, we created a plugin that generates an image based on the current canvas and submits this image to an outside service to be saved.

\subsection*{Restrictions}
While plugins were given a great amount of freedom when it came to interacting with the core elements, they are not allowed to modify Calico's interface. Plugins in particular can not create custom versions of \texttt{Scrap} elements or \texttt{Stroke} elements. We decided that we did not want to allow that flexibility. Instead, plugins are given just the freedom to manipulate elements, but not customize them further. Additionally, plugins are not allowed to customize the global interface of Calico -- they are not allowed to add buttons or change the layout of the menus. Again, these were not features we thought necessary for the level of customization that plugins were designed for.
\section{Input Handling}
% - talk about input handling system (how i know what object you touch)
% when the mouse pressed, locate all scraps that contain the mouse location
% determine the smallest scrap that contains the location
% actions are performed on that
% input handler is then "locked" onto that specific group (so that during the move, we are not trying to calculate over and over)
% unlocked as soon as it is released
% each object knows the list of points (or path) that acts as its bounds.
% Each object can then know if a specific point is within
In an effort to improve the user interaction experience, we felt that the existing input handling system needed to be recreated. The previous version of Calico had relied on the input handler provided by the drawing framework that was used (Piccolo \cite{piccolo}). This became unusable when working with canvases containing hundreds of objects. To improve response time, we needed to create a new input handler that did not rely on the drawing framework - meaning that it could continue to operate even while the drawing framework was busy rendering the display.

Rather than having a specific mouse listener linked to each object on screen, we decided to have a global mouse listener that determines which object is being touched. This means that mouse input only needs to be processed by a single handler, and based on various modes, can intelligently handle interaction with the user. 

Each canvas maintains a list of all objects that are present on screen. Each object also is responsible for maintaining a list of coordinates that form its ``bounds''. Objects then can readily determine if a specific point is contained within its ``bounds''.
Upon receiving mouse input, the input handler iterates through the elements and determines which elements contain the mouse location. This list is then further reduced to return only the \emph{smallest} object that contains the mouse location. This is helpful when handling scraps that have been stacked on top of each other - the parent scrap still contains the mouse location, but clearly the user wants to interact with one of the child elements.

After locating the element that the user is performing actions with, the input handler is ``locked'' to only interact with that element until the mouse has been released. This means that users can move a scrap around the screen (and move the mouse outside the bounds of the selected scrap) and still have the scrap follow the mouse location. Another benefit of locking the input handler on a specific element is that, while performing very intensive operations such as moving an element across the screen, we are not wasting resources to recalculate which element we are interacting with. This is something that was not available in the previous versions of Calico.

The one drawback to using this approach is that the object location and screen coordinate systems needed to be identical. What that means is that the Calico coordinate system is equal to the user's screen resolution. If a user with a larger screen joins a session, they can draw shapes and figures that are outside the viewable area of users with smaller screens. At the time, this was not a problem because all users were running the same resolution, so fortunately this problem rarely occurred.
\section{Administrative Interface}
% - admin web server
% uses httprequesthandlers
% velocity templates
% based off phpBB admin interface
% has a few basic functions, and allows admins to administer those
% view connected clients
% update configuration values
% upload images
% execute specific commands
% perform backup
% restore backups
One of the last components to be added was the administrative interface. Calico now had a headless server, and we needed a way to remotely manage it. We decided against building the interface within the Calico client as we wanted to keep the client as lightweight as possible. 
Instead, we created a web-based interface that allows the Calico server to be managed through a web browser.

The administrative interface only provides basic management abilities. The core abilities it provides are:
\begin{itemize}\itemsep1pt

\item
\textbf{Easily modify settings}.
We needed the ability to modify the settings of an existing Calico server. We created a page that lists all the configuration variables and lets the administrator modify those values. When saved, the Calico server can then be restarted to reflect the configuration changes.

\item
\textbf{View currently connected clients}. 
This is helpful in the classroom environment to be able to see all users who are currently connected to the server. In the future we plan to add the ability to ``kick'' clients from the server, but this was never full developed.

\item
\textbf{Perform backup and restore of sessions}.
One of the major requirements was the ability to be able to generate a backup of an entire session. Our goal was to provide the administrator with the ability to create a backup file that represented the current state of the server. This file needed to contain all the data required to fully restore all canvases and any scraps, strokes, or arrows that may be contained in those canvases. Once a backup file is created, we needed a way for a server to restore such a backup. Our administrative interface provides this functionality through a form where a backup file can be selected and then uploaded directly to a running server. The server then reads the contents of this file and import all data into the server. Restoring a backup essentially wipes any existing content and recreates the entire session from scratch. Once a restore has been performed, clients can reconnect and they can see all the data, just as it was when the backup was created.

\end{itemize}

These few requirements were the driving force behind the administrative interface. It stands as a very barebones and minimalistic interface that provides functions that we decided were essential to operation. This administrative interface has proved invaluable when generating and restoring backups. Having a backup of sessions provides users with a greater level of reassurance that any work they have made will not be lost should anything happen to the server.

% list of website paths and each one is mapped to a request handler for that specific action
% 
To keep the admin system lightweight, we decided to utilize the \texttt{HttpComponents} library provided by the Apache Foundation \cite{apache:http}. This library provides a single webserver endpoint that will route all web requests to a predefined list of request handler endpoints. These endpoints are specified in the \texttt{AdminRequestListenerThread} class within Calico. The most important part of this request handler is the ``request registry''. This registry acts as a routing engine that will route requests made to certain endpoints to a specific request handler designed to handle that request.

\begin{figure}[h!]
  \centering
  \small
  \verbatiminput{figures/java/request_registry.java}
  \normalsize
  \caption{Administration interface routing example}
  \label{code:request_registry}
\end{figure}

Figure \ref{code:request_registry} shows a code snippet of the routing table. The important part is the call to \texttt{reqistry.register}. In this snippet, you can see that any requests that are sent to \texttt{/gui/config/} will be sent to the \texttt{ConfigIndexRH} class to be processed. This allows us to break the administration system into logical components each designed to handle a different aspect of the server. 

\begin{figure}[h!]
  \centering
  \includegraphics[width=0.9\textwidth]{admin_ui.png}
  \caption{Admin configuration form (/gui/config/)}
  \label{fig:admin_config}
\end{figure}

One example request handler is the \texttt{ConfigIndexRH} class that was mentioned above. This class is responsible for two separate tasks. The first task is to process a \texttt{GET} request to \texttt{/gui/config/} which means that the configuration form should be displayed in the user's browser (an example form can be seen in Figure \ref{fig:admin_config}). To create the HTML required for the administration interface, we decided to use a templating framework that allows us to write HTML that contained placeholders that could be rendered by the request handlers. The template framework that we chose to use was Apache's Velocity \cite{velocity} framework. We decided to use Velocity because of its ease-of-use as well as powerful language that allowed us to create various macros to easily abstract common interface elements. The interface look-and-feel was copied from the administrative interface of an open-source forum system, PhpBB \cite{phpbb}. Figure \ref{code:tpl_config} shows a snippet of the template language that was used to generate the configuration form seen in Figure \ref{fig:admin_config}.

\begin{figure}[h!]
  \centering
  \small
  \verbatiminput{figures/tpl/config.tpl}
  \normalsize
  \caption{Snippet of the Velocity template for configuration page}
  \label{code:tpl_config}
\end{figure}

The second task of the configuration request handler was to process a \texttt{POST} request to the same endpoint. This request was for a form submission, which instructed the server to update the configuration settings on the server to reflect the values entered in the configuration form.

\chapter{Conclusion}
My work with Calico was geared toward solving the three problems found in the previous version of Calico. The first was the instability of Calico. Continual crashes and data loss plagued early versions. To solve this, we redesigned Calico to use a central server that would act as a stable storage system to prevent data loss. The second problem was the lack of support for collaboration. Our decision to move to a client-server architecture provided us with a solution to the problem of collaboration. Now with a centralized server, collaboration could be enabled by allowing multiple clients to connect and disambiguating concurrent edits to the same elements. The third problem was the lack of extensibility within Calico. Our creation of a well-defined plugin framework allows Calico to now be extended to meet future requirements with minimal change to the core.

Before these issues were resolved, Calico was used almost exclusively by our own design team for internal projects. By solving these problems, we were able to deploy Calico into many different environments such as professional workplaces and classrooms. We successfully tested Calico using a classroom of approximately 30 students working on various design projects, all connected to a single server. We also deployed Calico to a local software development company. At this local company, Calico was installed for a few weeks where it could be used by designers and developers on a daily basis. This study ended with mixed results, as Calico was not installed using a projector that was secured to the wall. This meant that the touch board continually needed to be re-calibrated, so the usage was not as consistent as it was in the classroom.

\section*{Experiences}
From these experiences, however, we learned...
add 2-3 paragraphs about experience

Not sure what should be in here, we never really ``tested'' the new iteration of calico (and I never studied any results). Should I talk about the use in the classroom, and that it was able to support 30+ students working on it at the same time? (I already talked about that in the previous paragraph)

\section*{Future Work}
While the work done to improve Calico was a major stepping stone, there are still many other directions for future work. One direction in particular would be the addition of sessions to Calico. Sessions would allow users to manage multiple server instances from a single administrative interface. This would allow a designer to create a new server that is completely separate from any existing instance (it would not share any canvases). Another feature that would be benefitial would be a subtype of scraps, called Lists. List objects would act just like a normal scrap element with one difference: they would allow content to be ordered within the list. This would allow designers to easily create ordered lists (or unordered) lists that they could easily refer to during development. 

% These commands fix an odd problem in which the bibliography line
% of the Table of Contents shows the wrong page number.
\clearpage
\phantomsection

% plain
%\bibliographystyle{abbrv} % DEFAULT
\bibliographystyle{acm}
%\bibliographystyle{abbrvnat}
\bibliography{thesis}

\appendix

\section{Calico Network Commands}
\begin{table}[h]
  \begin{tabular}{ | l | l | }
  \hline
  \textbf{Command Name} & \textbf{Format} \\ 
  \hline
  JOIN & SS \\
  \hline
  HEARTBEAT & LI \\
  \hline
  GROUP\_START & LLLI \\
  \hline
  GROUP\_APPEND & Lii \\
  \hline
  GROUP\_MOVE & LII \\
  \hline
  GROUP\_DELETE & L \\
  \hline
  GROUP\_DROP & L \\
  \hline
  GROUP\_FINISH & LB \\
  \hline
  GROUP\_SET\_CHILDREN & LII \\
  \hline
  GROUP\_SET\_PARENT & LL \\
  \hline
  GROUP\_MOVE\_START & L \\
  \hline
  GROUP\_MOVE\_END & LII \\
  \hline
  GROUP\_SET\_PERM & LI \\
  \hline
  GROUP\_RECTIFY & L \\
  \hline
  GROUP\_CIRCLIFY & L \\
  \hline
  GROUP\_CHILDREN\_COLOR & LIII \\
  \hline
  GROUP\_RELOAD\_START & LLLI \\
  \hline
  GROUP\_RELOAD\_FINISH & L \\
  \hline
  GROUP\_RELOAD\_COORDS & LIII \\
  \hline
  GROUP\_RELOAD\_CHILDREN & LIL \\
  \hline
  GROUP\_RELOAD\_POSITION & LII \\
  \hline
  GROUP\_RELOAD\_REMOVE & L \\
  \hline
  GROUP\_DUPLICATE & L \\
  \hline
  GROUP\_APPEND\_CLUSTER & LiII \\
  \hline
  GROUP\_SET\_CHILD\_GROUPS & Li \\
  \hline
  GROUP\_SET\_CHILD\_STROKES & Li \\
  \hline
  GROUP\_SET\_CHILD\_ARROWS & Li \\
  \hline
  GROUP\_REQUEST\_HASH\_CHECK & L \\
  \hline
  GROUP\_LOAD & LLLBiIIBddd \\
  \hline
  GROUP\_HASH\_CHECK & Li \\
  \hline
  GROUP\_COPY\_TO\_CANVAS & LLLII \\
  \hline
  GROUP\_SET\_TEXT & LS \\
  \hline
  GROUP\_SHRINK\_TO\_CONTENTS & L \\
  \hline
  GROUP\_IMAGE\_DOWNLOAD & LLSII \\
  \hline
  GROUP\_IMAGE\_LOAD & LLLSIIIIBiII \\
  \hline
  GROUP\_ROTATE & Ld \\
  \hline
  GROUP\_SCALE & Ldd \\
  \hline
  GROUP\_CREATE\_TEXT\_GROUP & LLSII \\
  \hline
  GROUP\_MAKE\_RECTANGLE & LIIII \\
  \hline
  GRID\_SIZE & II \\
  \hline
  UUID\_BLOCK & ILL \\
  \hline
  \end{tabular}
\end{table}

\begin{table}[h]
  \begin{tabular}{ | l | l | }
  \hline
  \textbf{Command Name} & \textbf{Format} \\ 
  \hline
  CANVAS\_INFO & LSII \\
  \hline
  CANVAS\_UPDATE & L \\
  \hline
  CANVAS\_UNDO & L \\
  \hline
  CANVAS\_REDO & L \\
  \hline
  CANVAS\_RELOAD\_START & L \\
  \hline
  CANVAS\_RELOAD\_FINISH & L \\
  \hline
  CANVAS\_RELOAD\_STROKES & LIL \\
  \hline
  CANVAS\_RELOAD\_GROUPS & LIL \\
  \hline
  CANVAS\_RELOAD\_ARROWS & LIL \\
  \hline
  CANVAS\_CLEAR\_FOR\_SC & L \\
  \hline
  CANVAS\_SC\_FINISH & L \\
  \hline
  CANVAS\_LOCK & LBSL \\
  \hline
  STATUS\_MESSAGE & S \\
  \hline
  ERROR\_MESSAGE & S \\
  \hline
  ERROR\_POPUP & S \\
  \hline
  CLICK\_TRACK & I \\
  \hline
  BGE\_APPEND & LII \\
  \hline
  BGE\_COLOR & LIII \\
  \hline
  BGE\_COORDS & LIII \\
  \hline
  BGE\_DELETE & L \\
  \hline
  BGE\_FINISH & L \\
  \hline
  BGE\_MOVE & LII \\
  \hline
  BGE\_START & LLL \\
  \hline
  BGE\_PARENT & LL \\
  \hline
  BGE\_CONSISTENCY & L \\
  \hline
  BGE\_RELOAD\_START & LLLIII \\
  \hline
  BGE\_RELOAD\_COORDS & LIII \\
  \hline
  BGE\_RELOAD\_FINISH & L \\
  \hline
  \end{tabular}
\end{table}

\begin{table}[h]
  \begin{tabular}{ | l | l | }
  \hline
  \textbf{Command Name} & \textbf{Format} \\ 
  \hline
  STROKE\_RELOAD\_START & LLLIII \\
  \hline
  STROKE\_RELOAD\_COORDS & LIII \\
  \hline
  STROKE\_RELOAD\_FINISH & L \\
  \hline
  STROKE\_RELOAD\_REMOVE & L \\
  \hline
  STROKE\_RELOAD\_POSITION & LII \\
  \hline

  STROKE\_START & LLLIII \\
  \hline
  STROKE\_APPEND & LiII \\
  \hline
  STROKE\_FINISH & L \\
  \hline
  STROKE\_SET\_COLOR & LIII \\
  \hline
  STROKE\_SET\_PARENT & LL \\
  \hline
  STROKE\_MOVE & LII \\
  \hline
  STROKE\_DELETE & L \\
  \hline
  STROKE\_LOAD & LLLCidddII \\
  \hline
  STROKE\_HASH\_CHECK & L \\
  \hline
  STROKE\_MAKE\_SCRAP & LL \\
  \hline
  STROKE\_MAKE\_SHRUNK\_SCRAP & LL \\
  \hline
  STROKE\_DELETE\_AREA & LL \\
  \hline
  STROKE\_ROTATE & Ld \\
  \hline
  STROKE\_SCALE & Ldd \\
  \hline
  STROKE\_SET\_AS\_POINTER & L \\
  \hline
  STROKE\_HIDE & LB \\
  \hline
  STROKE\_UNHIDE & L \\
  \hline

  ERASE\_START & L \\
  \hline
  ERASE\_END & LB \\
  \hline

  PLUGIN\_EVENT & S \\
  \hline



  CONSISTENCY\_CHECK &  \\
  \hline
  CONSISTENCY\_FINISH &  \\
  \hline
  CONSISTENCY\_CHECK\_CONTINUE & L \\
  \hline
  CONSISTENCY\_FAILED &  \\
  \hline
  CONSISTENCY\_RESYNC\_CANVAS & L \\
  \hline


  ARROW\_CREATE & LLICILIIILII \\
  \hline
  ARROW\_DELETE & L \\
  \hline
  ARROW\_SET\_TYPE & LI \\
  \hline
  ARROW\_SET\_COLOR & LIII \\
  \hline


  BACKUP\_FILE\_INFO & L \\
  \hline
  BACKUP\_FILE\_START &  \\
  \hline
  BACKUP\_FILE\_END &  \\
  \hline
  BACKUP\_FILE\_ATTR & SS \\
  \hline

  LIST\_CREATE & LLLLI \\
  \hline
  LIST\_LOAD & LLLBiII \\
  \hline
  LIST\_CHECK\_SET & LLLLB \\
  \hline
  \end{tabular}
\end{table}

Formats:
S-string. s-short. C-color. c-char. L-long. I-signed integer. i-unsigned integer. B-boolean. b-byte. f-float. d-double


\end{document}
